\section{古典概型}
古典概型是概率论发展初期的主要研究对象,它的一些概念直观简单,具有广泛应用。

\subsection{古典概型的定义}
\begin{BoxDefinition}[古典概型]
    若随机试验$E$满足以下两个特征
    \begin{enumerate}
        \item 试验的样本空间仅包含有限个元素。
        \item 试验中的每个基本事件发生的可能性相同。
    \end{enumerate}
    则将随机试验$E$称为\uwave{古典概型}(Classical Probability)或\uwave{等可能概型}。
\end{BoxDefinition}

\subsection{古典概型的概率公式}
\begin{BoxFormula}[古典概型的概率公式]
    若事件$A$包含$k$个基本事件,而样本空间的大小为$n$,则
    \begin{Equation}
        P(A)=\frac{k}{n}
    \end{Equation}
\end{BoxFormula}
\begin{Proof}
    设试验的样本空间为$S=\qty{e_1,e_2,\cdots,e_n}$,根据\fancyref{def:古典概型}
    \begin{Equation}&[1]
        P(\qty{e_1})=
        P(\qty{e_2})=
        \cdots
        P(\qty{e_n})
    \end{Equation}
    由于基本事件是两两互不相容的,且$S=\qty{e_1}\cup\qty{e_2}\cup\cdots\cup\qty{e_n}$
    \begin{Equation}&[2]
        1=P(S)=P\qty(\BigCup[i=1][n]\qty{e_i})=\Sum[i=1][n]P(\qty{e_i})
    \end{Equation}
    将\xrefpeq{1}代入\xrefpeq{2},这里$i$可以是任意小于等于$n$的正整数
    \begin{Equation}&[3]
        1=nP(\qty{e_i})
    \end{Equation}
    故
    \begin{Equation}
        P(\qty{e_i})=\frac{1}{n}
    \end{Equation}
    而事件$A$中包含$k$个基本试验,即$A=\qty{e_{a1},e_{a2},\cdots,e_{ak}}$
    \begin{Equation}*
        P(A)=\Sum[i=1][k]P(\qty{e_{ai}})=\Sum[i=1][k]\frac{1}{n}=\frac{k}{n}\qedhere
    \end{Equation}
\end{Proof}