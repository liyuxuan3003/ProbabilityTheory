\section{点估计优良性评判标准}
在\xref{sec:点估计}中我们看到,对于同一个参数,使用不同的估计方法求出的估计量可能是不同沟道,那么这时就有一个疑问,采用哪个估计量会更好些呢?这种优劣不是绝对的,而是基于某一评判标准而言的相对评价。这一节,我们会介绍三种常用的评判标准:无偏性、有效性、相合性。

\subsection{无偏性}
\begin{BoxDefinition}[无偏性]
    设$\hat{\theta}$是$\theta$的一个估计量,若对于任意的$\theta\in\Theta$,期望$E(\hat{\theta})$存在且满足
    \begin{Equation}
        E(\hat{\theta})=\theta
    \end{Equation}
    则称$\hat{\theta}$是$\theta$的\uwave{无偏估计}(Unbiased Estimator),否则称\uwave{有偏估计}(Biased Estimator)。
\end{BoxDefinition}
估计量的无偏性是指,由估计量得到的估计值相对于未知参数的真值来说,取某些样本观测值时偏大,取某些样本观测值时偏小,但是反复将这个估计量使用多次,就平均而言,偏差为零。估计若不具有无偏性,那无论使用多少次,其平均值也与真值有一定的距离,这个距离就是\uwave{系统误差}(Systematic Error)了,即由估计方法所决定的,无关测试随机性的固有误差。

根据\fancyref{fml:样本均值的期望}
\begin{Equation}
    E(\xbar{X})=\mu
\end{Equation}

根据\fancyref{fml:样本方差的期望}
\begin{Equation}
    E(S^2)=\sigma^2
\end{Equation}

这表明,样本均值$\xbar{X}$和样本方差$S^2$分别是$\mu$和$\sigma^2$的无偏估计。

那有什么估计量是有偏估计?最佳的实例就是与样本方差$S^2$相似的二阶中心矩$S_n^2$
\begin{Equation}
    S^2=\frac{1}{n-1}\Sum[k=1][n](X_k-\xbar{X})^2\qquad
    S_n^2=\frac{1}{n}\Sum[k=1][n](X_k-\xbar{X})^2
\end{Equation}
由于我们已经知道$E(S^2)=\sigma^2$
\begin{Equation}
    E(S_n^2)=\frac{n-1}{n}\sigma^2
\end{Equation}
因此$S_n^2$并非$\sigma^2$的无偏估计。不过若$n\to\infty$时$E(S_n^2)\to\sigma^2$,这表明$S_n^2$尽管有偏,但还不至于偏到很离谱的程度,随着$n$的增加偏差会逐渐减小,这类估计量称为\uwave{渐进无偏量}。

\subsection{有效性}
无偏估计并不是唯一的,可以有很多,如何在无偏估计中再进行选择?
\begin{BoxDefinition}[有效性]
    设$\hat{\theta_1},\hat{\theta_2}$是$\theta$的两个无偏估计,若对于任意的$\theta\in\Theta$
    \begin{Equation}
        D(\hat{\theta_1})\leq D(\hat{\theta_2})
    \end{Equation}
    且对于至少某一个$\theta\in\Theta$
    \begin{Equation}
        D(\hat{\theta_1})< D(\hat{\theta_2})
    \end{Equation}
    则称$\hat{\theta_1}$比$\hat{\theta_2}$更\uwave{有效}。
\end{BoxDefinition}

\subsection{相合性}
无偏性之上的要求是有效性,无偏性之下的要求则是相合性。
\begin{BoxDefinition}
    设$\hat{\theta}$是$\theta$的一个估计量,若对于任意的$\theta\in\Theta$有
    \begin{Equation}
        \forall\varepsilon>0,\quad\Lim[n\to\infty] P\qty{|\hat{\theta}-\theta|<\varepsilon}=1
    \end{Equation}
    则称$\hat{\theta}$是$\theta$的\uwave{相合估计}(Consistent Estimater)。
\end{BoxDefinition}
我们可以将相合性、无偏性、有效性的关系总结一下
\begin{itemize}
    \item 相合性是较低的要求,它要求估计量$\hat{\theta}$依概率收敛至$\theta$。
    \item 无偏性是较高的要求,它要求估计量$\hat{\theta}$的期望$E(\hat{\theta})$为$\theta$。
    \item 有效性则是在无偏的前提下,进一步讨论估计量$\hat{\theta}$的方差$D(\hat{\theta})$的大小。
\end{itemize}
这里尚有一个遗留的问题:依概率收敛与渐进无偏性之间是什么关系?有待思考。