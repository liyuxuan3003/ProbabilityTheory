\section{协方差与相关系数}
对于二维随机变量$(X,Y)$,除了讨论$X$与$Y$的期望和方差外,还需要讨论描述$X$与$Y$之间相互关系的数字特征,这是本节我们要讨论的重点,为此,将引入协方差和相关系数。

在\fancyref{ppt:随机变量和的方差}的证明中,我们看到,如果$X,Y$是相互独立的,则
\begin{Equation}
    E\qty\big{[X-E(X)][Y-E(Y)]}=0
\end{Equation}
这意味着当$E\qty\big{[X-E(X)][Y-E(Y)]}\neq 0$时,$X$与$Y$不是相互独立,而是存在联系的。

\begin{BoxDefinition}[协方差]
    设$X,Y$是随机变量,则称下式为$X,Y$的\uwave{协方差}(Covariance)
    \begin{Equation}
        \Cov(X,Y)=E\qty{[X-E(X)][Y-E(Y)]}
    \end{Equation}
\end{BoxDefinition}

很显然,协方差具有交换性,并且,随机变量与自身的协方差,就是方差。
\begin{BoxProperty}[协方差的交换性]
    协方差的定义具有交换性
    \begin{Equation}
        \Cov(X,Y)=\Cov(Y,X)
    \end{Equation}
\end{BoxProperty}

\begin{BoxProperty}[协方差的特例]
    协方差的特例是方差,同一随机变量与自身的协方差即为该随机变量的方差
    \begin{Equation}&[]
        \Cov(X,X)=D(X)\qquad
        \Cov(Y,Y)=D(Y)
    \end{Equation}
\end{BoxProperty}
\begin{Proof}
    根据\fancyref{def:方差}
    \begin{Equation}
        D(X)=E\qty{[X-E(X)]^2}\qquad
        D(Y)=E\qty{[Y-E(Y)]^2}
    \end{Equation}
    根据\fancyref{def:协方差}
    \begin{Equation}
        E\qty\big{[X-E(X)][Y-E(Y)]}=0
    \end{Equation}
    对比上两式即可得出\xrefpeq{}的结论。
\end{Proof}

协方差在一些线性性质方面具有与行列式类似的行为。
\begin{BoxProperty}[随机变量常数倍的协方差]
    若协方差中的两个随机变量分别乘了$a$倍$b$倍,则
    \begin{Equation}
        \Cov(aX,bY)=ab\Cov(X,Y)
    \end{Equation}
\end{BoxProperty}

\begin{BoxProperty}[随机变量和的协方差]
    若协方差中的某一项是两个随机变量的和,则
    \begin{Equation}
        \Cov(X_1+X_2,Y)=\Cov(X1,Y)+\Cov(X2,Y)
    \end{Equation}
\end{BoxProperty}

协方差$\Cov(x,y)$亦可以用均值或方差表示。
\begin{BoxFormula}[协方差的均值表示]
    协方差可以用均值表示为
    \begin{Equation}
        \Cov(X,Y)=E(XY)-E(X)E(Y)
    \end{Equation}
\end{BoxFormula}
\begin{Proof}
    这很容易证明,将\fancyref{def:协方差}展开
    \begin{Equation}
        \Cov(X,Y)=E\qty\big{[X-E(X)][Y-E(Y)]}=E\qty\big{XY-XE(Y)-YE(X)+E(X)E(Y)}
    \end{Equation}
    应用\fancyref{ppt:随机变量和的均值}
    \begin{Equation}*
        \Cov(X,Y)=E(XY)-E(X)E(Y)-E(X)E(Y)+E(X)E(Y)
    \end{Equation}
    化简
    \begin{Equation}*
        \Cov(X,Y)=E(XY)-E(X)E(Y)\qedhere
    \end{Equation}
\end{Proof}

\begin{BoxFormula}[协方差的方差表示]
    协方差可以用方差表示为
    \begin{Equation}
        \Cov(X,Y)=\frac{1}{2}[D(X+Y)-D(X)-D(Y)]
    \end{Equation}
\end{BoxFormula}

\begin{Proof}
    这也很容易证明,根据\fancyref{ppt:随机变量和的方差}
    \begin{Equation}
        D(X+Y)=D(X)+D(Y)+2E\qty{[X-E(X)][Y-E(Y)]}
    \end{Equation}
    而引入\fancyref{def:协方差}之后
    \begin{Equation}
        D(X+Y)=D(X)+D(Y)+2\Cov(X,Y)
    \end{Equation}
    因而
    \begin{Equation}*
        \Cov(x,y)=\frac{1}{2}[D(X+Y)-D(X)-D(Y)]\qedhere
    \end{Equation}
\end{Proof}

而与协方差相关的一个概念,称为相关系数,定义如下。
\begin{BoxDefinition}[相关系数]
    设$X,Y$是随机变量,则称下式为$X,Y$的\uwave{相关系数}(Correlation Coefficient)
    \begin{Equation}
        \rho_{XY}=\frac{\Cov(X,Y)}{\sqrt{D(x)}\sqrt{D(Y)}}
    \end{Equation}
\end{BoxDefinition}

那么相关系数有什么意义(其实也是在回答协方差有什么意义),且看下面的解释。
\begin{BoxDefinition}[均方误差]
    设$X,Y$是随机变量,现考虑以$X$的线性函数$a+bX$近似表示$Y$
    \begin{Equation}
        e=E\qty{[Y-(a+bX)]^2}
    \end{Equation}
    定义均方误差$e$表示以$a+bX$近似$Y$的好坏程度\footnote[2]{注意,均方误差在数学形式上更接近“方差$D(X)$”而不是“均方差$\sqrt{D(X)}$”。}。
\end{BoxDefinition}

很显然的是,当选取不同的$a,b$时,线性函数$aX+b$对$Y$的近似程度显然是不同的,故不同的$a,b$意味着不同的均方误差$e$。而下面的结论指出,相关系数$\rho_{XY}$与$e$的最小值有关。

\begin{BoxFormula}[相关系数与均方误差的关系]
    相关系数$\rho_{XY}$与不同$a,b$取值下均方误差$e$的最小值有关
    \begin{Equation}
        \min_{a,b}e=\min_{a,b}E\qty{[Y-(a+bX)]^2}=(1-\rho_{XY}^2)D(Y)
    \end{Equation}
\end{BoxFormula}
\begin{Proof}
    根据\fancyref{def:均方误差}
    \begin{Equation}&[1]
        e=E\qty{[Y-(a+bX)]^2}
    \end{Equation}
    展开
    \begin{Equation}&[2]
        e=E\qty{Y^2-2Y(a+bX)+(a+bX)^2}
    \end{Equation}
    再展开
    \begin{Equation}&[3]
        e=E\qty{Y^2-2aY-2bXY+a^2+2abX+b^2X^2}
    \end{Equation}
    应用\fancyref{ppt:随机变量和的均值},整理顺序
    \begin{Equation}&[4]
        e=E(Y^2)+b^2E(X^2)+a^2-2bE(XY)+2abE(X)-2aE(Y)
    \end{Equation}
    这里$e$是一个关于$a,b$的二元函数,其在$\pdv*{e}{a}=\pdv*{e}{b}=0$处取最小值。
    \begin{Gather}[10pt]
        \pdv{e}{a}=2a+2bE(X)-2E(Y)=0\xlabelpeq{4.1}\\
        \pdv{e}{b}=2bE(X^2)-2E(XY)+2aE(X)=0\xlabelpeq{4.2}
    \end{Gather}
    这样我们就可以确定$a$和$b$了,这其实是一个关于$a,b$的线性方程组
    \begin{Equation}&[5]
        \begin{pmatrix}
            1&E(X)\\
            E(X)&E(X^2)
        \end{pmatrix}
        \begin{pmatrix}
            a\\
            b
        \end{pmatrix}=
        \begin{pmatrix}
            E(Y)\\
            E(XY)
        \end{pmatrix}
    \end{Equation}
    计算系数行列式$D$,应用\fancyref{def:方差}
    \begin{Equation}&[6]
        D=
        \begin{vmatrix}
            1&E(X)\\
            E(X)&E(X^2)
        \end{vmatrix}=
        E(X^2)-E(X)^2=D(X)
    \end{Equation}
    计算系数行列式$D_a$
    \begin{Equation}&[7]
        D_a=
        \begin{vmatrix}
            E(Y)&E(X)\\
            E(XY)&E(X^2)
        \end{vmatrix}=
        E(X^2)E(Y)-E(X)E(XY)
    \end{Equation}
    这个式子不太好化简,我们一点点来,由于$D(x)=E(X^2)-E(X)^2$
    \begin{Equation}&[8]
        E(X^2)E(Y)=\qty[D(X)+E(X)^2]E(Y)
    \end{Equation}
    展开得
    \begin{Equation}&[9]
        E(X^2)E(Y)=D(X)E(Y)+E(X)^2E(Y)
    \end{Equation}\goodbreak
    将\xrefpeq{9}代入\xrefpeq{7}
    \begin{Equation}&[10]
        D_a=D(X)E(Y)+E(X)^2E(Y)-E(X)E(XY)
    \end{Equation}
    在\xrefpeq{10}的后两项中提出一个$E(X)$
    \begin{Equation}&[11]
        D_a=D(X)E(Y)+E(X)\qty\big[E(X)E(Y)-E(XY)]
    \end{Equation}
    根据\fancyref{def:协方差}
    \begin{Equation}&[12]
        D_a=D(X)E(Y)-E(X)\Cov(X,Y)
    \end{Equation}
    计算系数行列式$D_b$
    \begin{Equation}&[13]
        D_b=
        \begin{vmatrix}
            1&E(Y)\\
            E(X)&E(XY)
        \end{vmatrix}=
        E(XY)-E(X)E(Y)
    \end{Equation}
    根据\fancyref{def:协方差}
    \begin{Equation}&[14]
        D_b=\Cov(X,Y)
    \end{Equation}
    由此即得
    \begin{Equation}&[15]
        a_0=\frac{D_a}{D}=E(Y)-E(X)\frac{\Cov(X,Y)}{D(X)}\qquad
        b_0=\frac{D_b}{D}=\frac{\Cov(X,Y)}{D(X)}
    \end{Equation}
    这里解得的$a_0,b_0$就是使$e$最小时$a,b$的取值了。

    现在再让我们回到$e$上,重新从\xrefpeq{1}开始化简,依据$D(X)=E(X^2)-E(X)^2$
    \begin{Equation}&[16]
        \qquad\qquad
        \min_{a,b}e=E\qty{[Y-(a_0+b_0X)]^2}=
        D(Y-a_0-b_0X)+E(Y-a_0-b_0X)^2
        \qquad\qquad
    \end{Equation}
    根据\fancyref{ppt:常数的方差}
    \begin{Equation}&[17]
        \min_{a,b}e=D(Y-b_0X)+E(Y-a_0-b_0X)^2
    \end{Equation}
    而我们注意到
    \begin{Equation}&[18]
        E(Y-a_0-b_0X)=E(Y)-a_0-b_0E(X)
    \end{Equation}
    但同时,依据\xrefpeq{4.1},当$a=a_0, b=b_0$时
    \begin{Equation}&[19]
        \pdv{e}{a}=-2E(Y)+2a_0+2b_0E(X)=0
    \end{Equation}
    因此
    \begin{Equation}&[20]
        E(Y-a_0-b_0X)=-\frac{1}{2}\pdv{e}{a}=0
    \end{Equation}
    这样\xrefpeq{17}可以简化为
    \begin{Equation}&[21]
        \min_{a,b}e=D(Y-b_0X)
    \end{Equation}
    依据\fancyref{ppt:随机变量和的方差},根据\fancyref{ppt:随机变量常数倍的方差}
    \begin{Equation}&[22]
        \min_{a,b}e=D(Y)+b_0^2D(X)-2b_0\Cov(X,Y)
    \end{Equation}
    代入\xrefpeq{15}解出的$b_0=\Cov(X,Y)/D(X)$
    \begin{Equation}&[23]
        \min_{a,b}e=D(Y)+\frac{\Cov^2(X,Y)}{D(X)}-2\frac{\Cov^2(X,Y)}{D(X)}
    \end{Equation}
    即
    \begin{Equation}&[24]
        \min_{a,b}e=D(Y)-\frac{\Cov^2(X,Y)}{D(X)}
    \end{Equation}
    整理形式,代入\fancyref{def:相关系数}
    \begin{Equation}*
        \min_{a,b}e=D(Y)\qty[1-\frac{\Cov^2(X,Y)}{D(X)D(Y)}]=(1-\rho_{XY}^2)D(Y)\qedhere
    \end{Equation}
\end{Proof}

相关系数有两项较为重要的性质。
\begin{BoxProperty}[相关系数的取值范围]
    相关系数$\rho_{XY}$的取值范围是
    \begin{Equation}
        \abs{\rho_{XY}}\leq 1
    \end{Equation}
\end{BoxProperty}

\begin{Proof}
    根据\fancyref{fml:相关系数与均方误差的关系}
    \begin{Equation}
        \min_{a,b}e=\min_{a,b}E\qty{[Y-(a+bX)]^2}=(1-\rho_{XY}^2)D(Y)
    \end{Equation}
    而考虑到$D(Y)$的非负性,以及平方$\qty[Y-(a+bX)]^2$的均值的非负性
    \begin{Equation}
        1-\rho_{XY}^2=\frac{E\qty{\qty[Y-(a+bX)]^2}}{D(Y)}\geq 0
    \end{Equation}
    这就解得
    \begin{Equation}*
        \abs{\rho_{XY}}\leq 1\qedhere
    \end{Equation}
\end{Proof}

\begin{BoxProperty}[相关系数为一的充要条件]
    相关系数$\rho_{XY}=1$的充要条件是,存在常数$a^{*},b^{*}$,使得
    \begin{Equation}&[]
        P\qty{Y=a^{*}+b^{*}X}=1
    \end{Equation}
\end{BoxProperty}\goodbreak

\begin{Proof}
    \subparagraph{必要性}

    若已知$\abs{\rho_{XY}}=1$,则由\fancyref{fml:相关系数与均方误差的关系},存在$a_0,b_0$使得
    \begin{Equation}
        E\qty{[Y-(a_0+b_0X)]^2}=0
    \end{Equation}
    根据\fancyref{def:方差},$D(X)=E(X^2)-E(X)^2$
    \begin{Equation}
        D\qty\big[Y-(a_0+b_0X)]+E\qty\big[Y-(a_0+b_0X)]^2=0
    \end{Equation}
    考虑到两者的非负性,显然有
    \begin{Equation}
        D\qty\big[Y-(a_0+b_0X)]=0\qquad
        E\qty\big[Y-(a_0+b_0X)]=0
    \end{Equation}
    而根据\fancyref{ppt:方差为零的充要条件},由于$Y-(a_0+b_0X)$的方差为零
    \begin{Equation}
        P\qty\Big{Y-(a_0+b_0X)=E\big[Y-a_0+b_0X]}=1
    \end{Equation}
    即
    \begin{Equation}
        P\qty{Y-(a_0+b_0X)=0}=1
    \end{Equation}
    或
    \begin{Equation}
        P\qty{Y=a_0+b_0X}=1
    \end{Equation}

    \subparagraph{充分性}
    若存在常数$a^{*},b^{*}$使得
    \begin{Equation}
        P\qty{Y=a^{*}+b^{*}X}=1
    \end{Equation}
    即
    \begin{Equation}
        P\qty\big{Y-(a^{*}+b^{*}X)=0}=1
    \end{Equation}
    或
    \begin{Equation}
        P\qty{[Y-(a^{*}+b^{*}X)]^2=0}=1
    \end{Equation}
    若某个随机变量取某个值的概率为$1$,那么该随机变量的均值也是这个值,故
    \begin{Equation}
        E\qty{[Y-(a^{*}+b^{*}X)]^2}=0
    \end{Equation}
    但是另外一方面,根据\fancyref{fml:相关系数与均方误差的关系}
    \begin{Equation}
        E\qty{[Y-(a^{*}+b^{*}X)]^2}\geq\min_{a,b}E\qty{[Y-(a+bX)]^2}=(1-\rho_{XY}^2)D(Y)
    \end{Equation}
    因此
    \begin{Equation}
        (1-\rho_{XY}^2)D(Y)\leq 0
    \end{Equation}
    考虑到方差$D(Y)$的非负性
    \begin{Equation}
        (1-\rho_{XY}^2)\leq 0
    \end{Equation}
    故必有$\rho_{XY}=1$。
\end{Proof}

由此可见,均方误差的最小值$\mal{\min_{a,b}e}$是$\abs{\rho_{XY}}$的严格单调减少函数,因此
\begin{itemize}
    \item 当$|\rho_{XY}|$较大时$\mal{\min_{a,b}e}$较小,表明$X,Y$的线性相关程度较高。
    \item 当$|\rho_{XY}|$较小时$\mal{\min_{a,b}e}$较大,表明$X,Y$的线性相关程度较低。
\end{itemize}
特别的,两个极端情况是
\begin{itemize}
    \item 当$|\rho_{XY}|=1$时,在$X,Y$之间存在明确的线性关系,即$Y$总是可以表示为$a+bX$。
    \item 当$|\rho_{XY}|=0$时,在$X,Y$之间没有任何的线性关系,称为\uwave{不相关}。
\end{itemize}
特别需要注意的是相互独立和不相关的关系。\empx{相互独立可以推出不相关},因为在相互独立时有$E(XY)=E(X)E(Y)$,依据\xref{fml:协方差的均值表示},此时有$\Cov(X,Y)=E(XY)-E(X)E(Y)=0$,从而有$\rho_{XY}=0$。但反过来,\empx{不相关未必意味着相互独立},因为不相关时$\rho_{XY}=0$,通过同样的路径我们仍然能得到$\Cov(X,Y)=0$和$E(XY)=E(X)E(Y)$,而问题在于,尽管在独立时确实有$E(XY)=E(X)E(Y)$成立,但反过来,当$E(XY)=E(X)E(Y)$时未必能推出独立。\footnote{其实不太确定“不相关推不出相互独立”具体是不是卡在$E(XY)=E(X)E(Y)$与$X,Y$相互独立非充要上。}

直观上也可以理解,因为,不相关代表的是$X,Y$间不存在线性关系,不相关时,在$X,Y$间仍然可以存在其他关系,例如平方关系$X=Y^2$等。相互独立则代表$X,Y$没有任何的关系。

\begin{BoxProperty}[二维正态分布的协方差和相关系数]
    设随机变量服从二维正态分布,即$(X,Y)\sim N(\mu_1,\mu_2,\sigma_1^2,\sigma_2^2,\rho)$,则其协方差为
    \begin{Equation}
        \Cov(X,Y)=\rho\sigma_1\sigma_2
    \end{Equation}
    而相关系数为
    \begin{Equation}
        \rho_{XY}=\rho
    \end{Equation}
\end{BoxProperty}

\begin{Proof}
    根据\fancyref{def:协方差}
    \begin{Equation}&[1]
        \Cov(X,Y)=E\qty\big{[X-E(X)][Y-E(T)]}
    \end{Equation}
    在\fancyref{fml:二维正态分布的边缘分布}
    \begin{Equation}&[2]
        f_X(x)=\frac{1}{\sqrt{2\pi}\sigma_1}\exp[-\frac{(x-\mu_1)^2}{2\sigma_1^2}]\qquad
        f_Y(y)=\frac{1}{\sqrt{2\pi}\sigma_2}\exp[-\frac{(y-\mu_2)^2}{2\sigma_2^2}]
    \end{Equation}
    故知
    \begin{Equation}&[3]
        E(X)=\mu_1\quad E(Y)=\mu_2\quad
        D(X)=\sigma_1^2\quad
        D(Y)=\sigma_2^2
    \end{Equation}
    由此
    \begin{Equation}&[4]
        \Cov(X,Y)=\Int[-\infty][\infty]\Int[-\infty][\infty](x-\mu_1)(y-\mu_2)f(x,y)\dx\dy
    \end{Equation}
    而根据\fancyref{def:二维正态分布}
    \begin{Equation}&[5]
        f(x,y)=\frac{1}{2\pi\sigma_1\sigma_2\sqrt{1-\rho^2}}\exp\qty{
            \frac{-1}{2(1-\rho^2)}
            \qty[
            \frac{(x-\mu_1)^2}{\sigma_1^2}
            \hspace{-0.1em}-\hspace{-0.1em}
            2\rho\frac{(x\hspace{-0.1em}-\hspace{-0.1em}\mu_1)(x\hspace{-0.1em}-\hspace{-0.1em}\mu_2)}{\sigma_1\sigma_2}
            \hspace{-0.1em}+\hspace{-0.1em}
            \frac{(y-\mu_2)^2}{\sigma_2^2}]
        }
    \end{Equation}
    在\xrefpeq[二维正态分布的边缘分布]{5}中,我们曾证明$f(x,y)$可以化为
    \begin{Equation}&[6]
        \qquad
        f(x,y)=\frac{1}{2\pi\sigma_1\sigma_2\sqrt{1-\rho^2}}\exp\qty{
            \frac{-1}{2(1-\rho^2)}
            \qty(\frac{y-\mu_2}{\sigma_2}-\rho\frac{x-\mu_1}{\sigma_1})^2
            -\frac{(x-\mu_1)^2}{2\sigma_1^2}
        }
        \qquad
    \end{Equation}
    将\xrefpeq{6}代入\xrefpeq{4}
    \begin{Split}&[6.5]
        \Cov(X,Y)&=\frac{1}{2\pi\sigma_1\sigma_2\sqrt{1-\rho^2}}\Int[-\infty][\infty]\Int[-\infty][\infty]\\ &(x-\mu_1)(y-\mu_2)\exp\qty{
            \frac{-1}{2(1-\rho^2)}
            \qty(\frac{y-\mu_2}{\sigma_2}-\rho\frac{x-\mu_1}{\sigma_1})^2
            -\frac{(x-\mu_1)^2}{2\sigma_1^2}
        }\dx\dy
    \end{Split}

    若引入代换变量$u,v$
    \begin{Equation}&[7]
        u=\frac{x-\mu_1}{\sigma_1}\qquad
        v=\frac{1}{\sqrt{1-\rho^2}}\qty(\frac{y-\mu_2}{\sigma_2}-\rho\frac{x-\mu_1}{\sigma_1})
    \end{Equation}
    因而
    \begin{Equation}&[8]
        u^2=\frac{(x-\mu_1)^2}{\sigma_1^2}
    \end{Equation}
    以及
    \begin{Equation}&[9]
        uv=\frac{1}{\sqrt{1-\rho^2}}\qty[\frac{(x-\mu_1)(y-\mu_2)}{\sigma_1\sigma_2}-\rho\frac{(x-\mu_1)^2}{\sigma_1^2}]
    \end{Equation}
    从而
    \begin{Equation}&[10]
        \qquad\qquad
        \sigma_1\sigma_2\sqrt{1-\rho^2}uv+\rho\sigma_1\sigma_2u^2=(x-\mu_1)(y-\mu_2)-\rho\frac{\sigma_2}{\sigma_1}(x-\mu_1)^2+\rho\frac{\sigma_2}{\sigma_1}(x-\mu_1)^2
        \qquad\qquad
    \end{Equation}
    即
    \begin{Equation}&[11]
        \sigma_1\sigma_2\sqrt{1-\rho^2}uv+\rho\sigma_1\sigma_2u^2=(x-\mu_1)(y-\mu_2)
    \end{Equation}
    而同时
    \begin{Equation}&[12]
        \frac{u^2+v^2}{2}=
        \frac{1}{2(1-\rho^2)}
            \qty(\frac{y-\mu_2}{\sigma_2}-\rho\frac{x-\mu_1}{\sigma_1})^2
            +\frac{(x-\mu_1)^2}{2\sigma_1^2}
    \end{Equation}
    这样通过\xrefpeq{11}和\xrefpeq{12},\xrefpeq{6.5}中的被积函数就都可以由关于$x,y$转化为关于$u,v$了,但这里,其实真正的麻烦之处在于如何将$\dx\dy$换为$\dd{u}\dd{v}$,这需要涉及到雅可比行列式。

    首先由\xrefpeq{7}反解出$x,y$,由$u$的换元得到
    \begin{Equation}&[13]
        x=\sigma_1u+\mu_1
    \end{Equation}
    由$v$的换元得到
    \begin{Equation}&[14]
        \rho\frac{x-\mu_1}{\sigma_1}-\frac{y-\mu_2}{\sigma_2}=-\sqrt{1-\rho^2}v
    \end{Equation}
    两端同乘$\sigma_1\sigma_2$
    \begin{Equation}&[15]
        \rho\sigma_2(x-\mu_1)-\sigma_1(y-\mu_2)=-\sigma_1\sigma_2\sqrt{1-\rho^2}v
    \end{Equation}
    在\xrefpeq{15}中代入\xrefpeq{12}
    \begin{Equation}&[16]
        \rho\sigma_1\sigma_2u-\sigma_1y+\sigma_1\mu_2=-\sigma_1\sigma_2\sqrt{1-\rho^2}v
    \end{Equation}
    整理
    \begin{Equation}&[17]
        \sigma_1y=\rho\sigma_1\sigma_2u+\sigma_1\mu_2+\sigma_1\sigma_2\sqrt{1-\rho^2}v
    \end{Equation}
    得到
    \begin{Equation}&[18]
        y=\rho\sigma_2u+\mu_2+\sigma_2\sqrt{1-\rho^2}v
    \end{Equation}
    根据\xrefpeq{18}和\xrefpeq{12},计算雅可比式
    \begin{Equation}&[19]
        \pdv{(x,y)}{(u,v)}=
        \begin{pmatrix}
            \pdv*{x}{u}&\pdv*{x}{v}\\
            \pdv*{y}{u}&\pdv*{y}{v}    
        \end{pmatrix}=
        \begin{pmatrix}
            \sigma_1&0\\
            \rho\sigma_2&\sigma_2\sqrt{1-\rho^2}
        \end{pmatrix}=\sigma_1\sigma_2\sqrt{1-\rho^2}
    \end{Equation}
    因此
    \begin{Equation}&[20]
        \dx\dy=\abs{\pdv{(x,y)}{(u,v)}}\dd{u}\dd{v}=\sigma_1\sigma_2\sqrt{1-\rho^2}\dd{u}\dd{v}
    \end{Equation}
    让我们将\xrefpeq{20},\xrefpeq{12},\xrefpeq{11}代回\xrefpeq{6.5}
    \begin{Equation}&[21]
        \qquad\qquad
        \Cov(X,Y)=\frac{1}{2\pi}\Int[-\infty][\infty]\Int[-\infty][\infty]\qty(\sigma_1\sigma_2\sqrt{1-\rho^2}uv+\rho\sigma_1\sigma_2u)\e^{-(u^2+v^2)/2}\dd{u}\dd{v}
        \qquad\qquad
    \end{Equation}
    将积分拆分为两个部分
    \begin{Split}&[22]
        \qquad\qquad\quad
        \Cov(X,Y)=
        \frac{\sigma_1\sigma_2\sqrt{1-\rho^2}}{2\pi}\Int[-\infty][\infty]&\Int[-\infty][\infty]uv\e^{-(u^2+v^2)/2}\dd{u}\dd{v}
        \\[3mm]
        &+
        \frac{\rho\sigma_1\sigma_2}{2\pi}\Int[-\infty][\infty]
        \Int[-\infty][\infty]
        u^2\e^{-(u^2+v^2)/2}\dd{u}\dd{v}
        \qquad\qquad\quad
    \end{Split}
    转为累次积分
    \begin{Split}&[23]
        \qquad\qquad
        \Cov(X,Y)=
        \frac{\sigma_1\sigma_2\sqrt{1-\rho^2}}{2\pi}
        &\qty(\Int[-\infty][\infty]u\e^{-u^2/2}\dd{u})
        \qty(\Int[-\infty][\infty]v\e^{-v^2/2}\dd{v})\\[3mm]
        &+\frac{\rho\sigma_1\sigma_2}{2\pi}\qty(\Int[-\infty][\infty]u^2\e^{-u^2/2})\qty(\Int[-\infty][\infty]\e^{-v^2/2}\dd{v})
        \qquad\qquad
    \end{Split}
    这是四个高斯积分,第一行中的两个结果都是零,第二行的两个结果都是$\sqrt{2\pi}$
    \begin{Equation}
        \Cov(X,Y)=\frac{\rho\sigma_1\sigma_2}{2\pi}\sqrt{2\pi}\sqrt{2\pi}=\rho\sigma_1\sigma_2
    \end{Equation}
    最后,在根据\fancyref{def:相关系数}
    \begin{Equation}*
        \rho_{XY}=\frac{\Cov(X,Y)}{\sqrt{D(X)}\sqrt{D(Y)}}=\frac{\rho\sigma_1\sigma_2}{\sigma_1\sigma_2}=\rho\qedhere
    \end{Equation}
\end{Proof}

\fancyref{ppt:二维正态分布的协方差和相关系数}指出,二维正态分布的参数$\rho$的意义其实就是相关系数$\rho_{XY}$。另外,值得注意的是,在\xref{sec:二维随机变量的独立性}中,我们曾指出,二维正态分布中$X,Y$相互独立的充要条件是参数$\rho=0$。而$\rho=\rho_{XY}=0$也意味着$X,Y$不相关。这说明了一个重要的事实,\empx{在二维正态分布中,不相关和相互独立是等价的}。这是一般分布所不具有的性质。