\section{矩与协方差矩阵}

在前面几节中,我们依次学习了期望、方差、协方差等概念,尽管它们间的意义差异很大,但就数学形式而言,却有一定的相似性。期望是某个随机变量的均值,方差是某个随机变量减去其期望的平方的均值,协方差是两个随机变量减去各自期望的乘积的均值。在本节,我们将会介绍一种更一般的数字特征,称为\uwave{矩}或\uwave{动差}(Moment)。它是期望、方差、协方差的一般形式。

\begin{BoxDefinition}[原点矩]
    设$X,Y$是随机变量,若下式存在,则定义其为$X$的$k$阶\uwave{原点矩}(Raw Moment)
    \begin{Equation}
        E(X^k)
    \end{Equation}
    若下式存在,则定义其为$X$和$Y$的$k+l$阶\uwave{混合原点矩}
    \begin{Equation}
        E(X^kY^l)
    \end{Equation}
\end{BoxDefinition}

\begin{BoxDefinition}[中心矩]
    设$X,Y$是随机变量,若下式存在,则定义其为$X$的$k$阶\uwave{中心矩}(Central Moment)
    \begin{Equation}
        E\qty{[X-E(X)]^k}
    \end{Equation}
    若下式存在,则定义其为$X$和$Y$的$k+l$阶\uwave{混合中心矩}
    \begin{Equation}
        E\qty{[X-E(X)]^k[Y-E(Y)]^l}
    \end{Equation}
\end{BoxDefinition}\goodbreak

原先期望、方差、协方差的定义,现在均可以统一到矩的概念下来了
\begin{itemize}
    \item 随机变量$X$的期望$E(X)$是$X$的一阶原点矩。
    \item 随机变量$X$的方差$D(X)$是$X$的二阶中心矩。
    \item 随机变量$X,Y$的协方差$\Cov(X,Y)$是$X$和$Y$的二阶混合中心矩。
\end{itemize}

另一个与矩相关的概念是协方差矩阵,简单来说,协方差矩阵即是由二阶中心矩构成的矩阵。
\begin{BoxDefinition}[协方差矩阵]
    设$n$维随机变量$(X_1,X_2,\cdots,X_n)$的二阶中心矩都
    \begin{Equation}
        c_{ij}=\Cov(X_i,X_j)=E\qty{[X_i-E(X_i)][X_j-E(X_j)]}
    \end{Equation}
    都存在,则将矩阵
    \begin{Equation}
        \vb*{C}=
        \begin{pmatrix}
            c_{11}&c_{12}&\cdots&c_{1n}\\
            c_{21}&c_{22}&\cdots&c_{2n}\\
            \vdots&\vdots&\ddots&\vdots\\
            c_{n1}&c_{n2}&\cdots&c_{nn}\\
        \end{pmatrix}
    \end{Equation}
    称为$n$维随机变量$(X_1,X_2,\cdots,X_n)$的\uwave{协方差矩阵}(Covariance Matrix)。
\end{BoxDefinition}

特别的,对于$n=2$的协方差矩阵
\begin{Equation}
    \begin{pmatrix}
        c_{11}&c_{12}\\
        c_{21}&c_{22}
    \end{pmatrix}=
    \begin{pmatrix}
        D(X)&\Cov(X,Y)\\
        \Cov(Y,X)&D(Y)
    \end{pmatrix}
\end{Equation}

由于协方差满足交换律,因此协方差矩阵满足$c_{ij}=c_{ji}$,即,协方差矩阵是一个对称矩阵。


