\section{点估计}
设总体$X$的分布形式已知,但它的一个或多个参数未知。借助于总体$X$的一个样本来估计总体未知参数的问题,称为\uwave{点估计问题}(Point Estimation)。点估计问题的一般提法如下。
\begin{BoxDefinition}[点估计]
    设总体$X\sim f(x;\theta)$,其中$f$的形式已知,而$\theta$是未知参数,称为\uwave{总体参数}。

    总体参数$\theta$未知,但它的取值范围通常是已知的。
    
    总体参数$\theta$的取值范围为\uwave{参数空间},记作$\Theta$
    \begin{Equation}
        \theta\in\Theta
    \end{Equation}
    设$(X_1,X_2,\cdots,X_n)$是取自总体$X$的一个样本,若用一个统计量$\hat{\theta}$来估计$\theta$
    \begin{Equation}
        \hat{\theta}=\hat{\theta}(X_1,X_2,\cdots,X_n)
    \end{Equation}
    则称统计量$\hat{\theta}$是参数$\theta$的一个\uwave{点估计量}。
\end{BoxDefinition}

简单来说,点估计问题就是要为总体的分布参数$\theta$找到一个统计量$\hat{\theta}(X_1,X_2,\cdots,X_n)$进行估计,例如\xref{subsec:统计量的定义}中的$\xbar{X}=(1/n)\Sum[k=1][n]X_k$和$S^2=(1/n-1)\Sum[k=1][n](X_k-\xbar{X})^2$就是对总体的期望$\mu$和方差$\sigma^2$的点估计量,而且是一个不错的点估计量,当然,我们也可以胡乱定义一些关于$X_1,X_2,\cdots,X_n$的函数$L=\Sum[k=1][n]X_k^{42.1234}$作为$\mu$或$\sigma$的点估计量,这当然符合点估计量的定义,但后面我们引入对点估计的评价后就会看到,这显然是一个极糟的点估计量。

值得说明的是,若$\hat{\theta}$为$\theta$的估计量,则$g(\hat{\theta})$也是$g(\theta)$的估计量。

接下来,我们介绍两种系统的构造点估计量的手段:矩估计法和极大似然估计法。

\subsection{矩估计法}
矩估计法的核心就是替换思想:用样本原点矩$A_r$替换总体原点矩$\mu_r$。\footnote{关于$\mu_r$和$A_r$的定义,参见\fancyref{def:原点矩}和\fancyref{def:样本原点矩}。}当然,并非所有的参数$\theta$都恰好是总体原点矩$\mu_1,\mu_2,\cdots,\mu_m$中的某一个,但很多参数都可以可以表示为总体原点矩的函数,即$\theta=\phi(\mu_1,\mu_2,\cdots,\mu_m)$,矩估计法即用$\hat{\theta}=\phi(A_1,A_2,\cdots,A_m)$作为点估计量。

\begin{BoxDefinition}[矩估计法]
    设总体$X$的$r$阶原点矩为
    \begin{Equation}
        \mu_r=E(X^r)\qquad r=1,2,\cdots,m,\cdots
    \end{Equation}
    设总体$X$的样本$X_1,X_2,\cdots,X_n$的$r$阶样本原点矩为
    \begin{Equation}
        A_r=\frac{1}{n}\Sum[k=1][n]X_k^r\qquad r=1,2,\cdots,m,\cdots
    \end{Equation}
    若未知参数$\theta$可以表示为原点矩$\mu_1,\mu_2,\cdots,\mu_m$的函数
    \begin{Equation}
        \theta=\phi(\mu_1,\mu_2,\cdots,\mu_m)
    \end{Equation}
    则未知参数$\theta$的\uwave{矩估计量}为
    \begin{Equation}
        \hat{\theta}=\phi(A_1,A_2,\cdots,A_m)
    \end{Equation}
    这种点估计方法称为\uwave{矩估计法}(Method of Moments)。
\end{BoxDefinition}

接下来,我们来看一些具体的矩估计的例子。

\begin{BoxProperty}[二项分布的矩估计]
    设总体$X\sim B(1,p)$,其中$p$未知,证明$p$的点估计量为
    \begin{Equation}
        \hat{p}=A_1
    \end{Equation}
\end{BoxProperty}

\begin{Proof}
    根据\fancyref{ppt:二项分布的数值特征},对于二项分布$n=1$的特例0--1分布
    \begin{Equation}
        p=E(X)=\mu_1
    \end{Equation}
    根据\fancyref{def:矩估计法},将总体原点矩$\mu_1$替换为样本原点矩$A_1$
    \begin{Equation}*
        \hat{p}=A_1\qedhere
    \end{Equation}
\end{Proof}

\begin{BoxProperty}[泊松分布的矩估计]
    设总体$X\sim P(\lambda)$,其中$\lambda$未知,证明$\lambda$的点估计量可以表示为
    \begin{Equation}
        \hat{\lambda}=A_1\qquad
        \hat{\lambda}=A_2-A_1^2
    \end{Equation}
\end{BoxProperty}

\begin{Proof}
    根据\fancyref{ppt:泊松分布的数值特征},对于泊松分布
    \begin{Equation}
        \lambda=E(X)\qquad \lambda=D(X)=E(X^2)-E(X)^2
    \end{Equation}
    从期望出发,根据\fancyref{def:矩估计法}
    \begin{Equation}
        \hat{\lambda}=A_1
    \end{Equation}
    从方差出发,根据\fancyref{def:矩估计法}和\fancyref{fml:方差}
    \begin{Equation}
        \hat{\lambda}=A_2-A_1^2
    \end{Equation}
    这表明矩估计可能不唯一,这是矩估计的一个缺点,通常尽量采用较低阶的矩给出的估计。
\end{Proof}

\begin{BoxProperty}[均匀分布的矩估计]
    设总体$X\sim U(a,b)$,其中$a,b$未知,证明$a,b$的点估计量可以表示为
    \begin{Equation}
        \hat{a}=A_1-\sqrt{3(A_2-A_1^2)}\qquad
        \hat{b}=A_1+\sqrt{3(A_2-A_1^2)}
    \end{Equation}
\end{BoxProperty}
\begin{Proof}
    根据\fancyref{ppt:均匀分布的数值特征}
    \begin{Gather}[13pt]
        \mu_1=E(X)=\frac{(b+a)}{2}\xlabelpeq{1}\\
        \mu_2=E(X^2)=D(X)+E(X)^2=\frac{(b-a)^2}{12}+\frac{(b+a)^2}{4}\xlabelpeq{2}
    \end{Gather}
    由\xrefpeq{1}得到
    \begin{Equation}&[3]
        b+a=2\mu_1
    \end{Equation}
    由\xrefpeq{2}得到(代入\xrefpeq{1})
    \begin{Equation}&[4]
        b-a=2\sqrt{3(\mu_2-\mu_1^2)}
    \end{Equation}
    分别解出$a,b$
    \begin{Equation}
        a=\mu_1-\sqrt{3(\mu_2-\mu_1^2)}\qquad
        b=\mu_1+\sqrt{3(\mu_2-\mu_1^2)}
    \end{Equation}
    将$\mu_1,\mu_2$替换为$A_1,A_2$
    \begin{Equation}*
        \hat{a}=A_1-\sqrt{3(A_2-A_1^2)}\qquad
        \hat{b}=A_1+\sqrt{3(A_2-A_1^2)}\qedhere
    \end{Equation}
\end{Proof}

\begin{BoxProperty}[正态分布的矩估计]
    设总体$X\sim N(\mu,\sigma^2)$
    \begin{enumerate}
        \item 参数$\mu$的矩估计可以表示为
        \begin{Equation}
            \hat{\mu}=A_1
        \end{Equation}
        \item 若$\mu$已知,则参数$\sigma^2$的矩估计可以表示为
        \begin{Equation}
            \hat{\sigma^2}=A_2-\mu^2
        \end{Equation}
        \item 若$\mu$未知,则参数$\sigma^2$的矩估计可以表示为
        \begin{Equation}
            \hat{\sigma^2}=A_2-A_1^2
        \end{Equation}
    \end{enumerate}
\end{BoxProperty}

\begin{Proof}
    根据\fancyref{ppt:正态分布的数值特征},对于正态分布
    \begin{Equation}
        \mu=E(X)\qquad
        \sigma^2=D(X)=E(X^2)-E(X)^2
    \end{Equation}
    因此
    \begin{Equation}
        \hat{\mu}=A_1
    \end{Equation}
    若$\mu$已知,这里$E(X)$是已知的常数$\mu$,故
    \begin{Equation}
        \hat{\sigma^2}=A_2-\mu^2
    \end{Equation}
    若$\mu$未知,这里$E(X)$也要转换为$A_1$
    \begin{Equation}
        \hat{\sigma^2}=A_2-A_1^2
    \end{Equation}
    这表明,当分布具有多个参数时,一个参数是否已知会对另一参数的矩估计产生影响。
\end{Proof}

\fancyref{ppt:正态分布的矩估计}不仅适用于正态分布,也适用于一般分布\footnote{因为其证明中没有使用任何正态分布的特性,完全是基于均值和方差的定义。},由于$A_1=\xbar{X}$为样本均值,而$A_2-A_1^2=B_2$为二阶中心矩,也常记为$S_n^2$,因此当$\mu$和$\sigma^2$均未知时
\begin{itemize}
    \item 一阶原点矩$A_1=\xbar{X}$是期望$\mu$的矩估计量。
    \item 二阶中心矩$B_2=S_n^2$是方差$\sigma^2$的矩估计量。
\end{itemize}
矩估计法是一种很经典的估计方法,比较直观,计算简单,即便不知道总体分布类型,只需要知道未知参数和总体各阶原点矩的关系就能使用该方法,实际问题中,矩估计应用很广泛。

\subsection{极大似然估计法}
极大似然估计法是求总体未知参数的另一种常用的点估计方法,先通过一个示例来感受一下。
\begin{BoxExample}[极大似然估计法]
    设箱子中装有黑白两种颜色的球,一种颜色的有99个,一种颜色的1个,但不知道到底是哪种颜色是99个和1个。现从箱子中放回地抽取2个球,结果均是白色。

    现在问,哪种颜色的球有99个?哪种颜色的球有1个?
\end{BoxExample}

\begin{Solution}
    以最朴素的思想,我们会立即断言,既然抽了两个都是白球,那白球一定是多的那个。

    其实,这就是极大似然估计法的基本原理了。下面以数学的语言表述一下。

    设白球比例为$p$,根据条件,此处$p$只有$p=0.01$和$p=0.99$两种取值,但是
    \begin{itemize}
        \item 若$p=0.01$,则取得两个球都是白球的概率为$p^2=0.01^2$。
        \item 若$p=0.99$,则取得两个球都是白球的概率为$p^2=0.99^2$。
    \end{itemize}
    极大似然估计法的思想是,既然一件事已经发生了,那么我们就推定相关参数的取值一定会使得该事件发生的概率最大。很明显,这里取$p=0.99$将会使取到两个白球的概率最大。

    因此,极大似然估计法对这里的参数$p$给出的点估计即$\hat{p}=0.99$。
\end{Solution}

而一般来说,使得样本发生概率最大的概率$L$是一个关于未知参量$\theta$的连续函数$L(\theta)$,所以说,极大似然估计法所给出的点估计值$\hat{\theta}$,其实就是使得函数$L(\theta)$取极大值的那一$\theta$值。

\begin{BoxDefinition}[极大似然估计法]
    设总体$X$有分布律$p(x;\theta)$或密度函数$f(x;\theta)$,定义\uwave{似然函数}(Likelihood Function)
    \begin{Equation}
        L(\theta)=\Prod[k=1][n]p(X_k=x_k;\theta)\qquad
        L(\theta)=\Prod[k=1][n]f(x_k;\theta)
    \end{Equation}
    并称满足以下关系式
    \begin{Equation}
        L(\hat{\theta})=\max_{\theta\in\Theta} L(\theta)
    \end{Equation}
    的解$\hat{\theta}$为参量$\theta$的\uwave{极大似然估计量}。

    这种点估计方法称为\uwave{极大似然估计法}(Maximum Likelihood Estimation, MLE)。
\end{BoxDefinition}

当$L(\theta)$是可微函数时,求导是求极大似然估计最常用的方法,又因为$L(\theta)$和$\ln L(\theta)$z总是在同一个$\theta$处取到极值,且对数似然函数$\ln L(\theta)$的求导更简单,故常采用对数似然方程。
\begin{BoxEquation}[对数似然方程]
    当$L(\theta)$可微时,极大似然估计量$\hat{\theta}$常由\uwave{对数似然方程}求得
    \begin{Equation}
        \dv{\theta}\ln L(\theta)=0
    \end{Equation}
\end{BoxEquation}

现在让我们用极大似然估计法研究一下正态分布。

\begin{BoxProperty}[正态分布的极大似然估计]
    设总体$X\sim N(\mu,\sigma^2)$,若$\mu,\sigma^2$均未知,则两者的极大似然估计为
    \begin{Equation}
        \hat{\mu}=\xbar{X}\qquad
        \hat{\sigma^2}=S_n^2
    \end{Equation}
\end{BoxProperty}

\begin{Proof}
    根据\fancyref{def:正态分布}
    \begin{Equation}&[1]
        f(x;\mu,\sigma^2)=\frac{1}{\sqrt{2\pi}\sigma}\exp[-\frac{(x-\mu)^2}{2\sigma^2}]
    \end{Equation}
    根据\fancyref{def:极大似然估计法},似然函数$L(\mu,\sigma^2)$即将$x=x_k$的N个$f(x)$乘起来
    \begin{Equation}&[2]
        \qquad\qquad\qquad
        L(\mu,\sigma^2)=\Prod[k=1][n]f(x_k;\mu,\sigma^2)=\frac{1}{(\sqrt{2\pi}\sigma)^2}\exp[-\frac{1}{2\sigma^2}\Sum[k=1][n](x_k-\mu)^2]
        \qquad\qquad\qquad
    \end{Equation}
    或
    \begin{Equation}&[3]
        L(\mu,\sigma^2)=(2\pi)^{-n/2}(\sigma^2)^{-n/2}\exp[-\frac{1}{2\sigma^2}\Sum[k=1][n](x_k-\mu)^2]
    \end{Equation}
    在\xrefpeq{2}两端取对数(提醒下,$\ln\e^x=x$,$\ln a^x=x\ln a$)
    \begin{Equation}&[4]
        \ln L(\mu,\sigma^2)=-\frac{n}{2}\ln(2\pi)-\frac{n}{2}\ln(\sigma^2)-\frac{1}{2\sigma^2}\Sum[k=1][n](x_k-\mu)^2
    \end{Equation}
    这里$L(\mu,\sigma^2)$是二元函数,故需要求偏导。

    将$\ln L$对$\mu$求偏导
    \begin{Equation}&[5]
        \pdv{\mu}\ln L=-\frac{1}{2\sigma^2}\Sum[k=1][n]\pdv{\mu}\qty(x_k^2-2\mu x_k+\mu^2)=-\frac{1}{2\sigma^2}\Sum[k=1][n](-2x_k+2\mu)
    \end{Equation}
    即得
    \begin{Equation}&[6]
        \pdv{\mu}\ln L=\frac{1}{\sigma^2}\qty[\Sum[k=1][n](x_k)-n\mu]=0
    \end{Equation}
    将$\ln L$对$\sigma^2$求偏导
    \begin{Equation}&[7]
        \pdv{(\sigma^2)}\ln L=-\frac{n}{2\sigma^2}+\frac{1}{2(\sigma^2)^2}\Sum[k=1][n](x_k-\mu)^2=0
    \end{Equation}
    由\xrefpeq{6}解得
    \begin{Equation}&[8]
        \mu=\frac{1}{n}\Sum[k=1][n]x_k=\bar{x}
    \end{Equation}
    由\xrefpeq{7}解得(代入\xrefpeq{8})
    \begin{Equation}&[9]
        \sigma^2=\frac{1}{n}\Sum[k=1][n](x_k-\bar{x})^2
    \end{Equation}
    将\xrefpeq{8}和\xrefpeq{9}中的$\bar{x}$和$x_k$替换为$\xbar{X}$和$X_k$
    \begin{Equation}*
        \hat{\mu}=\xbar{X}\qquad
        \hat{\sigma^2}=\frac{1}{n}\Sum[k=1][n](X_k-\xbar{X})^2=S_n^2\qedhere
    \end{Equation}
\end{Proof}
该例表明,正态分布参量的极大似然估计量与矩估计量相同。但对于其他的分布则未必。

\begin{BoxProperty}[均匀分布的极大似然估计]
    设总体$X\sim U(a,b)$,若$a,b$均未知,则两者的极大似然估计为
    \begin{Equation}
        \hat{a}=\min X_k\qquad
        \hat{b}=\max X_k
    \end{Equation}
\end{BoxProperty}

\begin{Proof}
    根据\fancyref{def:均匀分布}
    \begin{Equation}
        f(x;a,b)=\begin{cases}
            \mal{\frac{1}{b-a}},&a\leq x\leq b\\[3mm]
            0,&\text{otherwise}
        \end{cases}
    \end{Equation}
    这里的似然函数其实特别简单,只有当$x_1,x_2,\cdots,x_n$均落在$a,b$之间时似然函数$L(a,b)$才有非零值$1/(b-a)^n$,否则$L(a,b)=0$。因此就有(有点怪?别忘了$L(a,b)$是$a,b$的函数!)
    \begin{Equation}
        L(a,b)=\begin{cases}
            \mal{\frac{1}{b-a}},&a\leq \min x_k,~ \max x_k\leq b\\[3mm]
            0,&\text{otherwise}
        \end{cases}
    \end{Equation}
    很明显,只有当$a=\min x_k$与$b=\max x_k$时,$L(a,b)$才能取得最大值
    \begin{Equation}
        a=\min x_k\qquad b=\max x_k
    \end{Equation}
    将$x_k$替换为$X_k$
    \begin{Equation}*
        \hat{a}=\min X_k\qquad
        \hat{b}=\max X_k\qedhere
    \end{Equation}
\end{Proof}

\fancyref{ppt:均匀分布的极大似然估计}的结论其实很直观,若已知某个随机变量服从均匀分布,现取得了一些样本,我们会将下限$a$和上限$b$分别估计为样本中的最小值和最大值。而通过与\fancyref{ppt:均匀分布的矩估计}的对比可以看出,矩估计和极大似然估计对均匀分布的$a,b$给出的点估计是不同的,这并不值得奇怪,不同的估计方法给出不同的结果,很合理。