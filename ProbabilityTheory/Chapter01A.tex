\section{随机试验}
我们遇到过各种试验,在这里,我们把试验作为一个含义广泛的术语,例如
\begin{itemize}
    \item $E_1$:将一枚硬币抛掷一次,观察正面$H$和反面$T$出现的情况。
    \item $E_2$:将一枚硬币抛掷三次,观察正面$H$和反面$T$出现的情况。 
    \item $E_3$:抛一颗骰子,观察出现的点数。
    \item $E_4$:在一批灯泡中任意抽取一只,测试它的极限寿命(以小时为单位)。
\end{itemize}
我们这里举了四个试验的例子,它们有一些共同的特点,我们总结如下
\begin{BoxDefinition}[随机试验]
    \uwave{随机试验}(Experiment)是指具有下述三个特点的试验,常记为$E$
    \begin{enumerate}
        \item 可以在相同的条件下重复进行。
        \item 试验的可能结果不止一个,并且事先能明确试验的所有可能结果。
        \item 试验进行之前,我们无法确定哪一个结果会出现
    \end{enumerate}
\end{BoxDefinition}

