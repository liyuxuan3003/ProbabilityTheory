\section{正态总体的区间估计}
本节将应用\xref{sec:区间估计}中的方法,就正态总体的均值$\mu$与方差$\sigma^2$作区间估计。

\subsection{均值的置信区间}
\begin{BoxProperty}[均值在方差已知时的置信区间]
    对于正态分布,当$\sigma^2$已知时,均值$\mu$的双侧$1-\alpha$置信区间为
    \begin{Equation}
        \qty\Bigg[\xbar{X}\pm\frac{\sigma}{\sqrt{n}}z_{\alpha/2}]
    \end{Equation}
    其两个单侧$1-\alpha$置信区间为\footnote[2]{提示:$z_{\alpha}<z_{\alpha/2}$。}
    \begin{Equation}
        \Biggl(-\infty~,~\xbar{X}+\frac{\sigma}{\sqrt{n}}z_\alpha\Biggr]\qquad
        \Biggl[\xbar{X}-\frac{\sigma}{\sqrt{n}}z_\alpha~,~+\infty\Biggr)        
    \end{Equation}
    其中$z_{\alpha}$是标准正态分布$N(0,1)$的上$\alpha$分位数。
\end{BoxProperty}

\begin{Proof}
    根据\fancyref{fml:样本均值的期望},即选取$\xbar{X}$作为均值$\mu$的无偏估计量
    \begin{Equation}
        E(\xbar{X})=\mu
    \end{Equation}
    根据\fancyref{thm:样本均值在已知方差时的分布},选取枢轴量$W$为
    \begin{Equation}
        W=\frac{\xbar{X}-\mu}{\sigma/\sqrt{n}}\sim N(0,1)
    \end{Equation}
    设$a,b$使得
    \begin{Equation}
        P\qty{a\leq\frac{\xbar{X}-\mu}{\sigma/\sqrt{n}}\leq b}=1-\alpha
    \end{Equation}
    这里$a,b$的解并不唯一,但习惯上我们希望$a,b$是对称区间。由于要求$W$分布在$[a,b]$中的概率是$1-\alpha$,故$W$分布在$a$左侧和$b$右侧的概率分别为$\alpha/2$,这即上$\alpha/2$分位数的定义
    \begin{Equation}
        P\qty{-z_{\alpha/2}\leq\frac{\xbar{X}-\mu}{\sigma/\sqrt{n}}\leq z_{\alpha/2}}=1-\alpha
    \end{Equation}
    由此即解得双侧$1-\alpha$的置信区间为
    \begin{Equation}
        \qty\Bigg[\xbar{X}-\frac{\sigma}{\sqrt{n}}z_{\alpha/2}~,~\xbar{X}+\frac{\sigma}{\sqrt{n}}z_{\alpha/2}]
    \end{Equation}
    类似的也可以讨论单侧置信区间的情况。
\end{Proof}

\begin{BoxProperty}[均值在方差未知时的置信区间]
    对于正态分布,当$\sigma^2$未知时,均值$\mu$的双侧$1-\alpha$置信区间为
    \begin{Equation}
        \qty\Bigg[\xbar{X}\pm\frac{S}{\sqrt{n}}t_{\alpha/2}(n-1)]
    \end{Equation}
    其两个单侧$1-\alpha$置信区间为
    \begin{Equation}
        \Biggl(-\infty~,~\xbar{X}+\frac{S}{\sqrt{n}}t_\alpha(n-1)\Biggr]\qquad
        \Biggl[\xbar{X}-\frac{S}{\sqrt{n}}t_\alpha(n-1)~,~+\infty\Biggr)        
    \end{Equation}
    其中$t_\alpha(n-1)$是$n-1$自由度的学生氏分布$t(n-1)$的上$\alpha$分位数。
\end{BoxProperty}

\begin{Proof}
    根据\fancyref{fml:样本均值的期望},即选取$\xbar{X}$作为均值$\mu$的无偏估计量
    \begin{Equation}
        E(\xbar{X})=\mu
    \end{Equation}
    根据\fancyref{thm:样本均值在未知方差时的分布},选取枢轴量$W$为
    \begin{Equation}
        W=\frac{\xbar{X}-\mu}{S/\sqrt{n}}\sim t(n-1)
    \end{Equation}
    设$a,b$使得
    \begin{Equation}
        P\qty{a\leq\frac{\xbar{X}-\mu}{S/\sqrt{n}}\leq b}=1-\alpha
    \end{Equation}
    完全类似的
    \begin{Equation}
        P\qty{-t_{\alpha/2}(n-1)\leq\frac{\xbar{X}-\mu}{S/\sqrt{n}}\leq t_{\alpha/2}(n-1)}=1-\alpha
    \end{Equation}
    由此即解得双侧$1-\alpha$的置信区间为
    \begin{Equation}
        \qty\Bigg[\xbar{X}-\frac{S}{\sqrt{n}}t_{\alpha/2}(n-1)~,~\xbar{X}+\frac{S}{\sqrt{n}}t_{\alpha/2}(n-1)]
    \end{Equation}
    类似的也可以讨论单侧置信区间的情况。
\end{Proof}

\subsection{方差的置信区间}
实际中,方差$\sigma^2$未知时通过均值$\mu$也是未知的,因此我们仅讨论这一种情况。

\begin{BoxProperty}[方差在均值未知时的执行区间]
    对于正态分布,当$\mu$未知时,方差$\sigma^2$的双侧$1-\alpha$置信区间为
    \begin{Equation}
        \qty\Bigg[\frac{(n-1)S^2}{\chi_{\alpha/2}^2(n-1)}~,~\frac{(n-1)S^2}{\chi_{1-\alpha/2}^2(n-1)}]
    \end{Equation}
    其两个单侧$1-\alpha$置信区间为
    \begin{Equation}
        \Biggl(-\infty~,~\frac{(n-1)S^2}{\chi_{1-\alpha}^2(n-1)}\Biggr]\qquad
        \Biggl[\frac{(n-1)S^2}{\chi_{\alpha}^2(n-1)}~,~+\infty\Biggr)        
    \end{Equation}
    其中$t_\alpha(n-1)$是$n-1$自由度的学生氏分布$t(n-1)$的上$\alpha$分位数。
\end{BoxProperty}

\begin{Proof}
    根据\fancyref{fml:样本方差的期望},即选取$S^2$作为方差$\sigma^2$的无偏估计量
    \begin{Equation}
        E(S^2)=\sigma^2
    \end{Equation}
    根据\fancyref{thm:样本方差的分布},选取枢轴量$W$为
    \begin{Equation}
        W=\frac{(n-1)S^2}{\sigma^2}\sim\chi^2(n-1)
    \end{Equation}
    设$a,b$使得
    \begin{Equation}
        P\qty{a\leq\frac{(n-1)S^2}{\sigma^2}\leq b}=1-\alpha
    \end{Equation}
    卡方分布$\chi^2(n-1)$较为特别,不同于标准正态分布$N(0,1)$和学生氏分布$t(n-1)$,卡方分布并不是对称的分布,因此$a,b$不可能选取为一个对称区间。那$a,b$如何确定呢?我们仍可以要求$W$分布在$a$左侧和$b$右侧的概率分别为$\alpha/2$,即$a,b$之外的概率是对称的,故
    \begin{Equation}
        P\qty{\chi_{1-\alpha/2}^2(n-1)\leq\frac{(n-1)S^2}{\sigma^2}\leq\chi_{\alpha/2}^2(n-1)}
    \end{Equation}
    由此即解得双侧$1-\alpha$的置信区间为
    \begin{Equation}
        \qty\Bigg[\frac{(n-1)S^2}{\chi_{\alpha/2}^2(n-1)}~,~\frac{(n-1)S^2}{\chi_{1-\alpha/2}^2(n-1)}]
    \end{Equation}
    类似的也可以讨论单侧置信区间的情况。
\end{Proof}