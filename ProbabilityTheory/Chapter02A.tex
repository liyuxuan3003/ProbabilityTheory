\section{随机变量}
让我们回到那个抛硬币的试验,假设抛三次,则样本空间为
\begin{Equation}*
    S=\qty{HHH, HHT, HTH, THH, HTT, THT, TTH, TTT}
\end{Equation}
以$X$记三次投掷得到正面$H$的总数,那么,对于样本空间$S=\qty{e}$中的每一个样本点$e$,它都可以对应$X$的一个数值,这样一来$X$就是一个定义在样本空间$S$上的实值单值函数
\begin{Equation}
    X=X(e)=\begin{cases}
        3,&e=HHH\\
        2,&e=HHT, HTH, THH\\
        1,&e=HTT\hspace{0.1em}, THT\hspace{0.1em}, TTH\\
        0,&e=TTT
    \end{cases}
\end{Equation}
函数$X$的定义域是样本空间$\qty{e}$,函数$X$的值域是$\qty{1,2,3,4}$。

\begin{BoxDefinition}[随机变量]
    设随机试验的样本空间为$S=\qty{e}$,而
    \begin{Equation}
        X=X(e)
    \end{Equation}
    是定义在$S$上的实值单值函数,则称$X=X(e)$为\uwave{随机变量}(Random Variable)。
\end{BoxDefinition}

简而言之,所谓随机变量,就是指将随机试验的每个样本点对应一个实数所构成的函数,当我们引入随机变量后,我们关于概率论的研究,就可以着眼于函数值,也就是随机变量的分布规律了,至于说,当随机变量取$1,2,3,\cdots$时到底表示什么,是掷骰子投出的点数?是抛硬币正面朝上的次数?是取出白球的数量?这就不是很重要了,这些具体的表象都被转化为了数字。\goodbreak

关于“随机变量是函数”这一点,我们要认清两点
\begin{enumerate}
    \item 随机试验的结果和随机变量之间是有确定的函数关系的
    \item 随机变量本身的结果是不确定的
\end{enumerate}
而我们的研究,主要是着眼于随机变量本身的不确定性结果,其实并不关心前者的函数关系。