\section{频率与概率}
对于一个事件来说,它在一次试验中可能会发生,也可能不会发生。我们常常希望知道某些事件在一次试验中发生的可能性有多大,我们希望找到一个合适的数来表征可能性的大小。为此,首先引入频率,它描述了事件发生的频繁程度,进而引出表征可能性大小的数,即概率。

\subsection{频率}
\begin{BoxDefinition}[频率]*
    在相同条件下,进行$n$次试验,记$n_A$为其中事件$A$发生的次数。

    定义$n_A$为$A$的\uwave{频数}(Frequency),定义$A$的\uwave{频率}(Relative Frequency)为\footnote[2]{需要注意,虽然物理上frequency就是频率,但统计上frequency是指频数。}
    \begin{Equation}
        f_n(A)=\frac{n_\text{A}}{n}
    \end{Equation}
    其中$f_n(A)$代表的是$A$在进行$n$次试验时的频数。
\end{BoxDefinition}\goodbreak

很容易验证,频率应具有以下的性质
\begin{BoxProperty}[频率的性质]
    频率具有以下基本性质
    \begin{enumerate}
        \item 频率总是在$0$和$1$间,即$0\leq f_n(A)\leq 1$
        \item 频率对于必然事件而言总是$1$,即$f_n(S)=1$
        \item 若$A_1,A_2,\cdots,A_n$是两两不相容的事件,则
        \begin{Equation}
            f_n(A_1\cup A_2\cup\cdots\cup A_n)=f_n(A_1)+f_n(A_2)+\cdots +f_n(A_n)
        \end{Equation}
    \end{enumerate}
\end{BoxProperty}

频率越大,事件发生的越频繁。但是,频率对于一个试验来说并不是一个唯一的值,进行的试验次数$n$不同,或者进行试验次数$n$相同的两组试验,得到的频率可能都是不同的,因此用频率表示可能性是不合适的。然而大量试验证实,当重复试验的次数$n$逐渐增大趋于无穷大时,频率$f_n(A)$呈现出稳定性,逐渐稳定于某个常数。这种频率稳定性即通常所说的统计规律性,让试验重复大量次数,计算频率$f_n(A)$,以它表征事件$A$发生的可能性是非常合适的。

然而在实际中,我们不可能对每一个事件都做大量的试验,因此,为了理论研究的需要,我们从频率的稳定性和频率和性质出发,给出如下表征事件发生可能性大小的概率的定义。

\subsection{概率}
\begin{BoxDefinition}[概率]
    若对于随机试验$E$的每一事件$A$赋予一个实数$P(A)$,且集合函数$P(\cdot)$满足
    \begin{enumerate}
        \item \textbf{非负性}:对于每一个事件$A$,有$P(A)\geq 0$
        \item \textbf{规范性}:对于必然事件$S$,有$P(S)=1$
        \item \textbf{可列可加性}:设$A_1,A_2,\cdots$是两两互不相容的事件,则
        \begin{Equation}
            P\qty(\BigCup[k=1][\infty]A_k)=\Sum[k=1][\infty]P(A_k)
        \end{Equation}
    \end{enumerate}
    那么我们就将$P(A)$称为$A$的\uwave{概率}(Probability)。
\end{BoxDefinition}

\xref{def:概率}是概率的公理化定义,称为\uwave{柯尔莫果洛夫公理}(Kolmogorov Axioms)\cite{W1,W2},它非常正确,但确实没有给出更多概率的内涵,我们看不出这和可能性有任何关系。例如对于掷硬币这个事件,直观上应有$P(H)=0.5, P(T)=0.5$,但即便指定$P(H)=0.3, P(T)=0.7$,这也是完全符合概率的公理化定义的,那,岂不是乱套了!这里我们应当指出,\empx{概率和概率律是两件事情},概率的公理化定义只保证概率的取值满足一些最基本的要求,比如硬币抛到正面的概率不能是$3.14$或$-273.15$,而至于说,概率具体是如何分布的,这其实是概率律的问题。\goodbreak

概率律的确立有很多方法,我们上述认为硬币抛出正反面的概率“显然”各为一半,其实就是在通过经验和统计数据建立概率论,而之后,我们还会学习如何通过严格的概率模型去建立概率律\cite{B2}。而结合严格的概率模型之后,我们就可以明确的证明概率是频率在试验次数趋于无穷大的极限(即所谓大数定律),从而确立概率表示可能性的内涵,暂且,先接受这一点。

概率的公理化定理的要求是最低的,例如,它事实上甚至都没有要求概率不能超过$1$,这是因为作为公理,要求总是应该越简单越好,这些很显然的性质其实可以通过三条公理演绎而来。

\begin{BoxProperty}[不可能事件的概率]
    不可能事件的概率为零
    \begin{Equation}
        P(\empty)=0
    \end{Equation}
\end{BoxProperty}

\begin{Proof}
    令$A_k=\empty, k=1,2,\cdots$,则显然
    \begin{Equation}
        \BigCup[k=1][\infty]A_k=\empty
    \end{Equation}
    且对于$\forall i,j\in\N^{*}$,若$i\neq j$,有$A_iA_j=\empty$,因此事件$A_k$之间两两相互独立。

    这样一来,根据\fancyref{def:概率}中概率的可列可加性
    \begin{Equation}
        P(\empty)=P\qty(\BigCup[k=1][\infty]A_k)=\Sum[k=1][\infty]P(A_k)=\Sum[k=1][\infty]P(\empty)
    \end{Equation}
    这就表明$P(\empty)$是无穷多个自身的和,根据\fancyref{def:概率}中概率的非负性,这里又应当有$P(\empty)\geq 0$,为了使上式成立,就必有$P(\empty)=0$,这就证明了不可能事件概率为零。
\end{Proof}

概率的有限可加性的实质,是将概率公理中的可列可加性,限制到有限个的结果。

\begin{BoxProperty}[概率的有限可加性]
    概率的有限可加性是指,若$A_1,A_2,\cdots,A_n$是两两互不相容的事件,则
    \begin{Equation}
        P\qty(\BigCup[k=1][n]A_k)=\Sum[k=1][n]P(A_k)
    \end{Equation}
\end{BoxProperty}



\begin{Proof}
    在$A_1,A_2,\cdots,A_n$的基础上,令$A_{n+1}=A_{n+2}=\cdots=\empty$,显然,
    \begin{Equation}
        \BigCup[k=1][\infty]A_k=
        \BigCup[k=1][n]A_k
    \end{Equation}
    现在$A_k$仍然满足两两互不相容,因此,可以利用\fancyref{def:概率}中概率的可列可加性
    \begin{Equation}
        P\qty(\BigCup[k=1][n]A_k)=
        P\qty(\BigCup[k=1][\infty]A_k)=
        \Sum[k=1][\infty]P(A_k)=\Sum[k=1][n]P(A_k)
    \end{Equation}
    这里最后一步应用了\fancyref{ppt:不可能事件的概率}的$P(\empty)=0$的结论。
\end{Proof}

\begin{BoxProperty}[包含事件的概率]
    设$A,B$是两个事件,若$A\subset B$,则有
    \begin{Equation}
        P(B-A)=P(B)-P(A)
    \end{Equation}
    并且
    \begin{Equation}
        P(B)>P(A)
    \end{Equation}
\end{BoxProperty}

\begin{Proof}
    运用文氏图很容易想象,如果$A\subset B$,那么
    \begin{Equation}
        B=A\cup (B-A)
    \end{Equation}
    并且$A$与$(B-A)$还是互斥的
    \begin{Equation}
        A(B-A)=\empty
    \end{Equation}
    这样就可以应用\fancyref{ppt:概率的有限可加性}
    \begin{Equation}
        P(B)=P(A)+P(B-A)
    \end{Equation}
    移项即得
    \begin{Equation}
        P(B-A)=P(B)-P(A)
    \end{Equation}
    根据\fancyref{def:概率}中概率的非负性,此处$P(B-A)>0$,故$P(B)>P(A)$。
\end{Proof}

\begin{BoxProperty}[概率的上限]
    对于任一事件$A$,总有
    \begin{Equation}
        P(A)\leq 1
    \end{Equation}
\end{BoxProperty}

\begin{Proof}
    由于$A\subset S$,根据\fancyref{ppt:包含事件的概率}和\fancyref{def:概率}的规范性
    \begin{Equation}*
        P(A)\leq P(S)=1\qedhere
    \end{Equation}
\end{Proof}

\begin{BoxProperty}[概率的加法公式]
    对于任意两事件$A,B$,总有
    \begin{Equation}
        P(A\cup B)=P(A)+P(B)-P(AB)
    \end{Equation}
\end{BoxProperty}

\begin{Proof}
    运用文氏图很容易想象,总有
    \begin{Equation}
        A\cup B=A\cup(B-AB)
    \end{Equation}
    并且$A$与$(B-A)$还是互斥的
    \begin{Equation}
        A(B-AB)=\empty
    \end{Equation}
    这样就可以应用\fancyref{ppt:概率的有限可加性}
    \begin{Equation}
        P(A\cup B)=P(A)+P(B-AB)
    \end{Equation}
    这里再应用\fancyref{ppt:包含事件的概率},考虑到$AB\subset B$
    \begin{Equation}*
        P(A\cup B)=P(A)+P(B)-P(AB)\qedhere
    \end{Equation}
\end{Proof}

\begin{BoxProperty}[逆事件的概率]
    对于任一事件$A$,总有
    \begin{Equation}
        P(\bar{A})=1-P(A)
    \end{Equation}
\end{BoxProperty}

\begin{Proof}
    根据\fancyref{def:对立事件},事件$A$和逆事件$\bar{A}$间满足
    \begin{Equation}
        A\cup\bar{A}=S\qquad
        A\bar{A}=\empty
    \end{Equation}
    即$A,\bar{A}$互不相容,故可运用\fancyref{def:可列可加性}
    \begin{Equation}
        P(S)=P(A\cup\bar{A})=P(A)+P(\bar{A})
    \end{Equation}
    而另外一方面,根据\fancyref{def:概率}中概率的规范性
    \begin{Equation}
        P(S)=1
    \end{Equation}
    因此
    \begin{Equation}*
        P(A)+P(\bar{A})=P(S)\qedhere
    \end{Equation}
\end{Proof}