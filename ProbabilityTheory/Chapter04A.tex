\section{随机变量的均值}

\subsection{均值的定义}
\begin{BoxDefinition}[均值]
    设离散型随机变量$X$具有分布律$p_k$,若级数$\Sum[k=1][\infty]x_kp_k$绝对收敛,则定义
    \begin{Equation}
        E(X)=\Sum[k=1][\infty]x_kp_k
    \end{Equation}

    设连续型随机变量$X$具有概率密度$f(x)$,若积分$\Int[-\infty][\infty]xf(x)\dx$绝对收敛,则定义
    \begin{Equation}
        E(X)=\Int[-\infty][\infty]xf(x)\dx
    \end{Equation}

    在上两式中,$E(X)$分别称为离散型和连续型随机变量$X$的\uwave{均值}(Mean)。
\end{BoxDefinition}

均值有时也被称为\uwave{数学期望}(Mathematical Expectation)或\uwave{期望}(Expectation),但是,这两个名称都不如“均值”来的浅显易懂。尽管我们或许会觉得$E(X)$的表达式是一些怪异的级数和积分,看似和平均值毫无关系,但仔细观察一下,就会发现$E(x)$的算式其实是在计算这样一种加权平均,统计量即随机变量的每一取值,统计量的权值即随机变量在相应取值处的概率或者概率密度,两者相乘后求和或积分,结果反映的不就是随机变量的平均状态嘛?

均值的定义中有一个细节,即要求计算加权平均的级数或积分绝对收敛,但是为什么一定要绝对收敛,不能仅仅是收敛呢?简而言之\cite{W5},均值$E(x)$作为刻画“平均值”这一概念的数学定义,是具有其客观意义的,应当符合我们对“平均值”这个概念的朴素认识。而在我们的认知中,改变值的次序不应当改变“平均值”的结果。然而在微积分3中我们就曾学过,级数只有在绝对收敛时才能任意改换求和顺序而不改变结果,级数若只是条件收敛,那通过构造不同的求和顺序,我们就可以得到不同的,甚至是任意所需的“平均值”了,这显然是不合适的。

\subsection{均值的性质}
接下来我们来讨论几个均值的重要性质,均以连续性为例予以证明。
\begin{BoxProperty}[常数的均值]
    设$C$是常数,则有
    \begin{Equation}
        E(C)=C
    \end{Equation}
\end{BoxProperty}
\begin{Proof}
    显而易见。
\end{Proof}

\begin{BoxProperty}[随机变量常数倍的均值]
    设$X$是随机变量,而$C$是常数,则有
    \begin{Equation}
        E(CX)=CE(X)
    \end{Equation}
\end{BoxProperty}
\begin{Proof}
    显而易见。
\end{Proof}

\begin{BoxProperty}[随机变量和的均值]
    设$X,Y$是随机变量,则有
    \begin{Equation}
        E(X+Y)=E(X)+E(Y)
    \end{Equation}
\end{BoxProperty}
\begin{Proof}
    设$(X,Y)$的概率密度为$f(x,y)$,边缘概率密度为$f_X(X),f_Y(y)$,根据\fancyref{def:均值}
    \begin{Equation}
        E(X+Y)=\Int[-\infty][\infty]\Int[-\infty][\infty](x+y)f(x,y)\dx\dy
    \end{Equation}
    拆分为两项
    \begin{Equation}
        E(X+Y)=
        \Int[-\infty][\infty]
        \Int[-\infty][\infty]
        xf(x,y)\dx\dy+
        \Int[-\infty][\infty]
        \Int[-\infty][\infty]
        yf(x,y)\dx\dy
    \end{Equation}
    根据\fancyref{def:边缘概率密度}
    \begin{Equation}
        E(X+Y)=
        \Int[-\infty][\infty]xf_X(x)\dx+
        \Int[-\infty][\infty]yf_Y(y)\dy
    \end{Equation}
    这里$f_X(x),f_Y(y)$即$X,Y$各自的概率密度了,根据\fancyref{def:均值}
    \begin{Equation}*
        E(X+Y)=E(X)+E(Y)\qedhere
    \end{Equation}
\end{Proof}

\fancyref{ppt:随机变量常数倍的均值}和\fancyref{ppt:随机变量和的均值}告诉我们,均值是一种线性运算。换言之,这表明,随机变量线性组合的均值,等于随机变量均值的线性组合。\vspace{3ex}

\begin{BoxProperty}[随机变量积的均值]
    设$X,Y$是随机变量,且相互独立,则有
    \begin{Equation}
        E(XY)=E(X)E(Y)
    \end{Equation}
\end{BoxProperty}

\begin{Proof}
    设$(X,Y)$的概率密度为$f(x,y)$,边缘概率密度为$f_X(X),f_Y(y)$,根据\fancyref{def:均值}
    \begin{Equation}&[1]
        E(XY)=\Int[-\infty][\infty]\Int[-\infty][\infty]xyf(x,y)\dx\dy
    \end{Equation}
    根据\fancyref{thm:连续型随机变量的独立性},由于$X,Y$相互独立
    \begin{Equation}&[2]
        f(x,y)=f_X(x)f_Y(y)
    \end{Equation}
    将\xrefpeq{2}代入\xrefpeq{1}
    \begin{Equation}&[3]
        E(XY)=\Int[-\infty][\infty]\Int[-\infty][\infty]xyf_X(x)f_Y(y)\dx\dy
    \end{Equation}
    将$x,y$的积分分离
    \begin{Equation}&[4]
        E(XY)=
        \qty[\Int[-\infty][\infty]xf_X(x)\dx]
        \qty[\Int[-\infty][\infty]yf_Y(y)\dy]
    \end{Equation}
    这里$f_X(x),f_Y(y)$即$X,Y$各自的概率密度了,根据\fancyref{def:均值}
    \begin{Equation}*
        E(XY)=E(X)E(Y)\qedhere
    \end{Equation}
\end{Proof}

\fancyref{ppt:随机变量积的均值}中需要注意,该性质只能适用于相互独立的随机变量。