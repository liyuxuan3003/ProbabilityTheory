\section{条件概率}
我们说,假如有100人进行抽签,比如说有2个签或之类的,虽然我们知道,先抽和后抽中签的概率并没有差异,均是$1/100$,但是在等待抽签的过程中,总有这样朴素的想法
\begin{itemize}
    \item 如果上来就有人抽到了签,那么之后我再中签的概率就小很多
    \item 如果上来很多人都没有抽到前,那么之后我再中签的概率就大很多
\end{itemize}
这种朴素的想法,在概率论中称为\uwave{条件概率}。

\subsection{条件概率的定义}
\begin{BoxDefinition}[条件概率]
    设$A,B$是两个事件,且$P(A)>0$,那么将
    \begin{Equation}
        P(B\mid A)=\frac{P(AB)}{P(A)}
    \end{Equation}
    称为在事件$A$发生的条件下,事件$B$发生的\uwave{条件概率}(Conditional Probability)。
\end{BoxDefinition}

事件$B$在事件$A$发生的前提下发生的条件概率为
\begin{Equation}
    P(B\mid A)=\frac{P(AB)}{P(A)}
\end{Equation}
事件$B$发生的概率,其实可以认为是在必然事件$S$发生的前提下发生的条件概率
\begin{Equation}
    P(B)=P(B\mid S)=\frac{P(SB)}{P(S)}
\end{Equation}
条件概率在文氏图中的表示常常是一个挺让人困惑的问题,因为$P(AB)$和$P(B\mid A)$分别代表“$A,B$同时发生的概率”和“$A$发生后$B$发生的概率”,在文氏图上好像都应用$A,B$的交集来表示,但很明显两者又不相等,这怎么解释呢?其实关键在于,样本空间的划定范围,同样是$A,B$的交集,计算$P(AB)$时的样本空间是$S$,计算$P(B\mid A)$时的样本空间则是$A$。

条件概率$P(\cdot\mid A)$,容易验证,亦满足\fancyref{def:概率}的三条概率公理。

\subsection{条件概率与乘法公式}
概率乘法公式,即求两个事件同时发生的概率,其实是\fancyref{def:条件概率}的简单改写。
\begin{BoxFormula}[概率的乘法公式]
    设$A,B$是两个事件,两者同时发生的概率为
    \begin{Equation}
        P(AB)=P(B\mid A)P(A)
    \end{Equation}
\end{BoxFormula}
概率乘法公式是容易理解的,$A,B$同时发生的概率,即$A$发生的概率乘以$A$发生时$B$发生的概率。这种想法也可以很容易的推广到$n$个事件的情形,若有$A_1,A_2,\cdots,A_n$共$n$个事件
\begin{Equation}
    \qquad\quad
    P(A_1A_2\cdots A_n)=P(A_n\mid A_1A_2\cdots A_{n-1})\cdots P(A_3\mid A_2A_1)P(A_2\mid A_1)P(A_1)
    \qquad\quad
\end{Equation}

\subsection{全概率公式和贝叶斯公式}
在介绍全概率公式和贝叶斯公式前,我们先来介绍样本空间的划分的定义。

\begin{BoxDefinition}[划分]
    设$S$为试验$E$的样本空间,$B_1,B_2,\cdots,B_n$为$E$的一组事件,若
    \begin{enumerate}
        \item $B_iB_j=\empty, i,j=1,2,\cdots, n, i\neq j$,即$B_1,B_2,\cdots,B_n$两两互不相容
        \item $B_1\cup B_2\cup\cdots\cup B_n=S$
    \end{enumerate}
    则称$B_1,B_2,\cdots,B_n$是样本空间的一个\uwave{划分}(Partition)。
\end{BoxDefinition}

简而言之,\empx{划分就是将一个集合拆成若干个完备但互不重叠的集合}。

\begin{BoxTheorem}[全概率定理]
    设$A$为$E$的事件,而$B_1,B_2,\cdots,B_n$为$S$的一个划分,且$P(B_i)>0$,则
    \begin{Equation}
        P(A)=\Sum[k=1][n]P(A\mid B_k)P(B_k)
    \end{Equation}
    该结论称为\uwave{全概率定理}(Law of Total Probability)。
\end{BoxTheorem}

\begin{Proof}
    根据\fancyref{def:划分}
    \begin{Equation}&[1]
        A=AS=A(B_1\cup B_2\cup\cdots\cup B_n)
    \end{Equation}
    运用交集(即乘法)对并集的分配律
    \begin{Equation}&[2]
        A=AB_1\cup AB_2\cup\cdots\cup AB_n
    \end{Equation}
    根据\fancyref{def:划分},我们知道$B_1,B_2,\cdots,B_n$两两互不相容,故$AB_1,AB_2,\cdots,AB_n$亦两两互不相容,因此,运用\xrefpeq{2}和\fancyref{ppt:概率的有限可加性},概率$P(A)$可以表示为
    \begin{Equation}&[3]
        P(A)=
        P\qty(\BigCup[k=1][n]AB_k)=\Sum[k=1][n]P(AB_k)
    \end{Equation}
    根据\fancyref{fml:概率的乘法公式}
    \begin{Equation}*
        P(A)=\Sum[k=1][n]P(A\mid B_k)P(B_k)\qedhere
    \end{Equation}
\end{Proof}

\begin{BoxTheorem}[贝叶斯定理]
    设$A$为$E$的事件,而$B_1,B_2,\cdots,B_n$为$S$的一个划分,且$P(A),P(B_i)>0$,则
    \begin{Equation}
        P(B_i\mid A)=\frac{P(A\mid B_i)P(B_i)}{P(A)}=\frac{P(A\mid B_i)P(B_i)}{\Sum[k=1][n]P(A\mid B_k)P(B_k)}
    \end{Equation}
    该结论称为\uwave{贝叶斯定理}(Bayes' Theorem)。
\end{BoxTheorem}
\begin{Proof}
    根据\fancyref{def:条件概率}
    \begin{Equation}
        P(B_i\mid A)=\frac{P(AB_i)}{P(A)}
    \end{Equation}
    在上式中,就$P(A)$代入\fancyref{thm:全概率定理},并使用\fancyref{fml:概率的乘法公式}
    \begin{Equation}*
        P(B_i\mid A)=\frac{P(A\mid B_i)P(B_i)}{\Sum[k=1][n]P(A\mid B_k)P(B_k)}\qedhere
    \end{Equation}
\end{Proof}

现在让我们来看一个应用贝叶斯定理和全概率定理的例子。

% \begin{BoxExample}[条件概率的计算实例1]*
%     设有两个袋子,甲和乙,两个袋子均装有若干小球
%     \begin{itemize}
%         \item 在甲袋中,装有$1$个白球和$2$个黑球。
%         \item 在乙袋中,装有$2$个白球和$1$个黑球。
%     \end{itemize}
%     设两个事件$A,B$
%     \begin{itemize}
%         \item 以$A$表示“从甲袋中取出一个白球”。
%         \item 以$B$表示“从乙袋中取出一个白球”。
%     \end{itemize}
%     现在试问,由乙袋中任取一个小球放入甲袋,再从甲袋中取出一个球,是白球的概率。
% \end{BoxExample}

% \begin{Proof}
%     根据\fancyref{thm:全概率定理}
%     \begin{Equation}
%         P(A)=P(A\mid B)P(B)+P(A\mid\bar{B})P(\bar{B})
%     \end{Equation}
%     换言之,从乙袋取出一个白球的概率,可以分为两部分计算
%     \begin{itemize}
%         \item $P(B)P(A\mid B)$即乙袋取出白球放入甲袋,从甲袋取出白球的概率。
%         \item $P(\bar{B})P(A\mid\bar{B})$即乙袋取出黑球放入甲袋,从甲袋取出白球的概率。
%     \end{itemize}
%     很明显
%     \begin{Equation}
%         P(B)=\frac{2}{3}\qquad
%         P(\bar{B})=\frac{1}{3}
%     \end{Equation}
%     如果乙袋放入甲袋的是白球,那么甲袋中,有$2$个白球和$2$个黑球
%     \begin{Equation}
%         P(A\mid B)=\frac{2}{4}
%     \end{Equation}
%     如果乙袋放入甲袋的是黑球,那么甲袋中,有$1$个白球和$3$个黑球
%     \begin{Equation}
%         P(A\mid\bar{B})=\frac{1}{4}
%     \end{Equation}
%     因此
%     \begin{Equation}*
%         P(A)=\frac{2}{4}\cdot\frac{2}{3}+\frac{1}{4}\cdot\frac{1}{3}=\frac{5}{12}\qedhere
%     \end{Equation}
% \end{Proof}

\begin{BoxExample}[条件概率的计算实例]*
    设有一种癌症,根据历史统计数据,在人群中患有该癌症的概率为$0.005$。

    设现在有一种癌症筛查手段,记
    \begin{itemize}
        \item 以$A$表示事件“被试者的反应为阳性”。
        \item 以$B$表示事件“被试者患有癌症”。
    \end{itemize}
    现在已知这种癌症筛查手段满足
    \begin{itemize}
        \item 若有癌症,阳性概率$P(A\mid B)=0.95$,即,真阳性率为$0.95$,假阴性率为$0.05$。
        \item 若无癌症,阴性概率$P(\bar{A}\mid \bar{B})=0.95$,即,真阴性率为$0.95$,假阳性率为$0.05$。
    \end{itemize}
    现讨论该癌症筛查手段的有效性,即检测为阳性,且患癌症的概率$P(B\mid A)$。
\end{BoxExample}

\begin{Solution}
    首先,人群中患有该癌症的概率是$0.005$,即
    \begin{Equation}&[1]
        P(B)=0.005
    \end{Equation}
    即
    \begin{Equation}&[1.5]
        P(\bar{B})=0.995
    \end{Equation}
    根据真阳性率为$0.95$的条件
    \begin{Equation}&[2]
        P(A\mid B)=0.95
    \end{Equation}
    根据假阳性率为$0.05$的条件
    \begin{Equation}&[3]
        P(A\mid\bar{B})=1-P(\bar{A}\mid\bar{B})=0.05
    \end{Equation}
    应用\fancyref{thm:全概率定理}
    \begin{Equation}&[4]
        P(A)=P(A\mid B)P(B)+P(A\mid\bar{B})P(\bar{B})
    \end{Equation}
    代入\xrefpeq{1},\xrefpeq{1.5},\xrefpeq{2},\xrefpeq{3}
    \begin{Equation}&[5]
        P(A)=0.95\cdot 0.005+0.05\cdot 0.995=0.0545
    \end{Equation}
    这表明,人群中进行筛查的阳性率(无论是真阳还是假阳)为$0.0545$。

    应用\fancyref{thm:贝叶斯定理}
    \begin{Equation}
        P(B\mid A)=\frac{P(BA)}{P(A)}
    \end{Equation}
    应用\fancyref{fml:概率的乘法公式}
    \begin{Equation}
        P(B\mid A)=\frac{P(A\mid B)P(B)}{P(A)}
    \end{Equation}
    即得
    \begin{Equation}*
        P(B\mid A)=\frac{0.95*0.005}{0.0545}=0.0872\qedhere
    \end{Equation}
\end{Solution}

这表明,尽管这种癌症筛查手段的假阳性率和假阴性率都只有$0.05$,但是如果发现检测为阳性,真正患癌症的概率只有$0.0872$,是很低的,这是为什么呢?实际上,关键在于,作为一个大规模筛查罕见病的手段,假阳性率应当要很低,因为健康人口的基数很大,这就导致会尽管假阳性率并不高,但出现的假阳性案例甚至会比真阳性还多的多,造成筛查的有效性被破坏。