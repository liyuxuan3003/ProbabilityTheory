\chapter{概率论的基本概念}
自然界中发生的现象是多种多样的,有这样一类现象,在一定条件下必然会发生,例如,向上抛一物体必然下落,同性电荷必然相互排斥,异性电荷必然相互吸引,等等,这类现象称为\uwave{确定性现象}(Deterministic Phenomenon)。但是,还有这样一类现象,例如,在相同条件下抛同一枚硬币,其落下后,可能朝上,可能朝下。这里现象,在一定的条件下,可能出现这样的结果,可能出现那样的结果,但是经过长期实践并深入研究后,发现这类现象在大量重复试验或观察下,它的结果却呈现出某种规律性,抛硬币得到正面朝上和反面朝上的情况数大概都是一半。这种在大量重复试验或观察中所呈现出的固有规律性,就是我们之后所说的\uwave{统计规律性}(Statistical Regularity)。综上,\empx{在个别试验中结果呈现出不确定性,在大量重复试验中结果又具有统计规律性的现象},我们称之为\uwave{随机现象}(Random Phenomenon)。而所谓\uwave{概率论}(Probability Theory)与\uwave{数理统计}(Mathematical Statistics),即研究统计规律的数学。

\section{随机试验}
我们遇到过各种试验,在这里,我们把试验作为一个含义广泛的术语,例如
\begin{itemize}
    \item $E_1$:将一枚硬币抛掷一次,观察正面$H$和反面$T$出现的情况。
    \item $E_2$:将一枚硬币抛掷三次,观察正面$H$和反面$T$出现的情况。 
    \item $E_3$:抛一颗骰子,观察出现的点数。
    \item $E_4$:在一批灯泡中任意抽取一只,测试它的极限寿命(以小时为单位)。
\end{itemize}
我们这里举了四个试验的例子,它们有一些共同的特点,我们总结如下
\begin{BoxDefinition}[随机试验]
    \uwave{随机试验}(Experiment)是指具有下述三个特点的试验,常记为$E$
    \begin{enumerate}
        \item 可以在相同的条件下重复进行。
        \item 试验的可能结果不止一个,并且事先能明确试验的所有可能结果。
        \item 试验进行之前,我们无法确定哪一个结果会出现
    \end{enumerate}
\end{BoxDefinition}


\section{随机事件与样本空间}

\subsection{样本空间和样本点}
关于\fancyref{def:随机试验}中提到的一个随机试验,尽管在每次试验前并不能预知试验的结果,但试验的所有可能组成的集合是已知的,这种集合,可以是有限集,可以是无限集。
\begin{BoxDefinition}[样本空间]
    定义随机试验$E$每个可能的结果组成的集合$S$,为$E$的\uwave{样本空间}(Sample Space)
    \begin{Equation}
        S=\qty{e_1,e_2,\cdots,e_i,\cdots}
    \end{Equation}
\end{BoxDefinition}
\begin{BoxDefinition}[样本点]
    定义随机试验$E$的每个结果$e_i$,为$E$的\uwave{样本点}(Sample Point)。
\end{BoxDefinition}

例如\xref{sec:随机试验}中列出的试验$E_1,E_2,E_2,E_3$的样本空间就是
\begin{Gather}*[6pt]
    S_1:\qty{H,T}\\
    S_2:\qty{HHH,HHT,HTH,HTT,THH,THT,TTH,TTT}\\
    S_3:\qty{1,2,3,4,5,6}\\
    S_4:\qty{t\mid t\geq 0}
\end{Gather}
而上述样本空间的集合中的每个元素,就是样本点。

\subsection{随机事件}
在实际中,当进行随机试验时,我们常常关心某些样本点所组成的集合。
\begin{BoxDefinition}[随机事件]
    \uwave{随机事件}(Event)是随机试验$E$的样本空间$S$的一个子集,常记为$A$

    在每次试验$E$中,当且仅当该子集$A$中一个样本点发生,我们称这一事件$A$发生。
\end{BoxDefinition}

例如,关于\xref{sec:随机试验}中掷骰子的随机试验$E_3$
\begin{itemize}
    \item 事件$A_1$是掷出点数$2$,则$A_1=\qty{2}$
    \item 事件$A_2$是掷出偶数点数,则$A_2=\qty{2,4,6}$
\end{itemize}
特别的,我们赋予那些仅包含一个样本点的事件一个特别的名称,即基本事件。
\begin{BoxDefinition}[基本事件]
    \uwave{基本事件}(Elementary Event)是指仅包含一个样本点的事件。
\end{BoxDefinition}
\begin{itemize}
    \item 掷骰子的试验$E_3$中,基本事件有六个,有$\qty{1},\qty{2},\qty{3},\qty{4},\qty{5},\qty{6}$
    \item 掷硬币的试验$E_1$中,基本事件有两个,有$\qty{H},\qty{T}$
\end{itemize}

% 除此之外,还有两个特别的事件需要定义
\begin{BoxDefinition}[必然事件]
    定义\uwave{必然事件}(Certain Event)为样本空间$S$自身,它是$S$的子集。
\end{BoxDefinition}
\begin{BoxDefinition}[不可能事件]
    定义\uwave{不可能事件}(Impossible Event)为空集$\empty$,它是$S$的子集。
\end{BoxDefinition}

\subsection{随机事件的关系与运算}
我们说,\empx{事件是一个集合},因此,事件间的关系和运算可以按照集合论中集合之间的关系和运算来处理,下面,我们将根据“事件发生”的含义,给出这些运算在概率论中的具体含义。

\begin{BoxDefinition}[事件的包含]
    若$A\subset B$,则称事件$B$包含事件$A$,即事件$A$发生必然导致事件$B$发生。
\end{BoxDefinition}

\begin{BoxDefinition}[事件的相等]
    若$A\subset B$又有$B\subset A$,则称事件$B$等于事件$A$,记为$A=B$。
\end{BoxDefinition}

\begin{BoxDefinition}[和事件]
    定义$A,B$的\uwave{和事件}(Union of Events)
    \begin{Equation}
        A\cup B=\qty{x\mid x\in A~\text{or}~x\in B}
    \end{Equation}
    其意义是,当且仅当$A,B$至少有一个发生时,和事件$A\cap B$发生。
\end{BoxDefinition}

\begin{BoxDefinition}[积事件]*
    定义$A,B$的\uwave{积事件}(Intersection of Events)
    \begin{Equation}
        A\cap B=AB=\qty{x\mid x\in A~\text{and}~x\in B}
    \end{Equation}
    其意义是,当且仅当$A,B$同时发生时,积事件$A\cup B$发生。
\end{BoxDefinition}

\begin{BoxDefinition}[差事件]
    定义$A,B$的\uwave{差事件}(Set Difference)
    \begin{Equation}
        A-B=\qty{x\mid x\in A~\text{and}~x\notin B}
    \end{Equation}
    其意义是,当且仅当$A$发生且$B$不发生,差事件$A-B$发生。
\end{BoxDefinition}

\begin{BoxDefinition}[互斥事件]
    若事件$A.B$满足下式,则$A,B$是\uwave{互斥}的
    \begin{Equation}
        A\cap B=\empty
    \end{Equation}
\end{BoxDefinition}

\begin{BoxDefinition}[对立事件]
    若事件$A.B$满足下式,则$A,B$是\uwave{对立}的
    \begin{Equation}
        A\cap B=\empty\qquad A\cup B=S
    \end{Equation}
\end{BoxDefinition}
很明显,对立事件是互斥事件的一种特殊情况,对立必然互斥,但对立还要求“除了$A,B$没有更多情况”,因此,我们可以将$A$唯一的那个对立事件记为$\bar{A}$,显然其满足$\bar{A}=S-A$。

\section{频率与概率}
对于一个事件来说,它在一次试验中可能会发生,也可能不会发生。我们常常希望知道某些事件在一次试验中发生的可能性有多大,我们希望找到一个合适的数来表征可能性的大小。为此,首先引入频率,它描述了事件发生的频繁程度,进而引出表征可能性大小的数,即概率。

\subsection{频率}
\begin{BoxDefinition}[频率]*
    在相同条件下,进行$n$次试验,记$n_A$为其中事件$A$发生的次数。

    定义$n_A$为$A$的\uwave{频数}(Frequency),定义$A$的\uwave{频率}(Relative Frequency)为\footnote[2]{需要注意,虽然物理上frequency就是频率,但统计上frequency是指频数。}
    \begin{Equation}
        f_n(A)=\frac{n_\text{A}}{n}
    \end{Equation}
    其中$f_n(A)$代表的是$A$在进行$n$次试验时的频数。
\end{BoxDefinition}\goodbreak

很容易验证,频率应具有以下的性质
\begin{BoxProperty}[频率的性质]
    频率具有以下基本性质
    \begin{enumerate}
        \item 频率总是在$0$和$1$间,即$0\leq f_n(A)\leq 1$
        \item 频率对于必然事件而言总是$1$,即$f_n(S)=1$
        \item 若$A_1,A_2,\cdots,A_n$是两两不相容的事件,则
        \begin{Equation}
            f_n(A_1\cup A_2\cup\cdots\cup A_n)=f_n(A_1)+f_n(A_2)+\cdots +f_n(A_n)
        \end{Equation}
    \end{enumerate}
\end{BoxProperty}

频率越大,事件发生的越频繁。但是,频率对于一个试验来说并不是一个唯一的值,进行的试验次数$n$不同,或者进行试验次数$n$相同的两组试验,得到的频率可能都是不同的,因此用频率表示可能性是不合适的。然而大量试验证实,当重复试验的次数$n$逐渐增大趋于无穷大时,频率$f_n(A)$呈现出稳定性,逐渐稳定于某个常数。这种频率稳定性即通常所说的统计规律性,让试验重复大量次数,计算频率$f_n(A)$,以它表征事件$A$发生的可能性是非常合适的。

然而在实际中,我们不可能对每一个事件都做大量的试验,因此,为了理论研究的需要,我们从频率的稳定性和频率和性质出发,给出如下表征事件发生可能性大小的概率的定义。

\subsection{概率}
\begin{BoxDefinition}[概率]
    若对于随机试验$E$的每一事件$A$赋予一个实数$P(A)$,且集合函数$P(\cdot)$满足
    \begin{enumerate}
        \item \textbf{非负性}:对于每一个事件$A$,有$P(A)\geq 0$
        \item \textbf{规范性}:对于必然事件$S$,有$P(S)=1$
        \item \textbf{可列可加性}:设$A_1,A_2,\cdots$是两两互不相容的事件,则
        \begin{Equation}
            P\qty(\BigCup[k=1][\infty]A_k)=\Sum[k=1][\infty]P(A_k)
        \end{Equation}
    \end{enumerate}
    那么我们就将$P(A)$称为$A$的\uwave{概率}(Probability)。
\end{BoxDefinition}

\xref{def:概率}是概率的公理化定义,称为\uwave{柯尔莫果洛夫公理}(Kolmogorov Axioms)\cite{W1,W2},它非常正确,但确实没有给出更多概率的内涵,我们看不出这和可能性有任何关系。例如对于掷硬币这个事件,直观上应有$P(H)=0.5, P(T)=0.5$,但即便指定$P(H)=0.3, P(T)=0.7$,这也是完全符合概率的公理化定义的,那,岂不是乱套了!这里我们应当指出,\empx{概率和概率律是两件事情},概率的公理化定义只保证概率的取值满足一些最基本的要求,比如硬币抛到正面的概率不能是$3.14$或$-273.15$,而至于说,概率具体是如何分布的,这其实是概率律的问题。\goodbreak

概率律的确立有很多方法,我们上述认为硬币抛出正反面的概率“显然”各为一半,其实就是在通过经验和统计数据建立概率论,而之后,我们还会学习如何通过严格的概率模型去建立概率律\cite{B2}。而结合严格的概率模型之后,我们就可以明确的证明概率是频率在试验次数趋于无穷大的极限(即所谓大数定律),从而确立概率表示可能性的内涵,暂且,先接受这一点。

概率的公理化定理的要求是最低的,例如,它事实上甚至都没有要求概率不能超过$1$,这是因为作为公理,要求总是应该越简单越好,这些很显然的性质其实可以通过三条公理演绎而来。

\begin{BoxProperty}[不可能事件的概率]
    不可能事件的概率为零
    \begin{Equation}
        P(\empty)=0
    \end{Equation}
\end{BoxProperty}

\begin{Proof}
    令$A_k=\empty, k=1,2,\cdots$,则显然
    \begin{Equation}
        \BigCup[k=1][\infty]A_k=\empty
    \end{Equation}
    且对于$\forall i,j\in\N^{*}$,若$i\neq j$,有$A_iA_j=\empty$,因此事件$A_k$之间两两相互独立。

    这样一来,根据\fancyref{def:概率}中概率的可列可加性
    \begin{Equation}
        P(\empty)=P\qty(\BigCup[k=1][\infty]A_k)=\Sum[k=1][\infty]P(A_k)=\Sum[k=1][\infty]P(\empty)
    \end{Equation}
    这就表明$P(\empty)$是无穷多个自身的和,根据\fancyref{def:概率}中概率的非负性,这里又应当有$P(\empty)\geq 0$,为了使上式成立,就必有$P(\empty)=0$,这就证明了不可能事件概率为零。
\end{Proof}

概率的有限可加性的实质,是将概率公理中的可列可加性,限制到有限个的结果。

\begin{BoxProperty}[概率的有限可加性]
    概率的有限可加性是指,若$A_1,A_2,\cdots,A_n$是两两互不相容的事件,则
    \begin{Equation}
        P\qty(\BigCup[k=1][n]A_k)=\Sum[k=1][n]P(A_k)
    \end{Equation}
\end{BoxProperty}



\begin{Proof}
    在$A_1,A_2,\cdots,A_n$的基础上,令$A_{n+1}=A_{n+2}=\cdots=\empty$,显然,
    \begin{Equation}
        \BigCup[k=1][\infty]A_k=
        \BigCup[k=1][n]A_k
    \end{Equation}
    现在$A_k$仍然满足两两互不相容,因此,可以利用\fancyref{def:概率}中概率的可列可加性
    \begin{Equation}
        P\qty(\BigCup[k=1][n]A_k)=
        P\qty(\BigCup[k=1][\infty]A_k)=
        \Sum[k=1][\infty]P(A_k)=\Sum[k=1][n]P(A_k)
    \end{Equation}
    这里最后一步应用了\fancyref{ppt:不可能事件的概率}的$P(\empty)=0$的结论。
\end{Proof}

\begin{BoxProperty}[包含事件的概率]
    设$A,B$是两个事件,若$A\subset B$,则有
    \begin{Equation}
        P(B-A)=P(B)-P(A)
    \end{Equation}
    并且
    \begin{Equation}
        P(B)>P(A)
    \end{Equation}
\end{BoxProperty}

\begin{Proof}
    运用文氏图很容易想象,如果$A\subset B$,那么
    \begin{Equation}
        B=A\cup (B-A)
    \end{Equation}
    并且$A$与$(B-A)$还是互斥的
    \begin{Equation}
        A(B-A)=\empty
    \end{Equation}
    这样就可以应用\fancyref{ppt:概率的有限可加性}
    \begin{Equation}
        P(B)=P(A)+P(B-A)
    \end{Equation}
    移项即得
    \begin{Equation}
        P(B-A)=P(B)-P(A)
    \end{Equation}
    根据\fancyref{def:概率}中概率的非负性,此处$P(B-A)>0$,故$P(B)>P(A)$。
\end{Proof}

\begin{BoxProperty}[概率的上限]
    对于任一事件$A$,总有
    \begin{Equation}
        P(A)\leq 1
    \end{Equation}
\end{BoxProperty}

\begin{Proof}
    由于$A\subset S$,根据\fancyref{ppt:包含事件的概率}和\fancyref{def:概率}的规范性
    \begin{Equation}*
        P(A)\leq P(S)=1\qedhere
    \end{Equation}
\end{Proof}

\begin{BoxProperty}[概率的加法公式]
    对于任意两事件$A,B$,总有
    \begin{Equation}
        P(A\cup B)=P(A)+P(B)-P(AB)
    \end{Equation}
\end{BoxProperty}

\begin{Proof}
    运用文氏图很容易想象,总有
    \begin{Equation}
        A\cup B=A\cup(B-AB)
    \end{Equation}
    并且$A$与$(B-A)$还是互斥的
    \begin{Equation}
        A(B-AB)=\empty
    \end{Equation}
    这样就可以应用\fancyref{ppt:概率的有限可加性}
    \begin{Equation}
        P(A\cup B)=P(A)+P(B-AB)
    \end{Equation}
    这里再应用\fancyref{ppt:包含事件的概率},考虑到$AB\subset B$
    \begin{Equation}*
        P(A\cup B)=P(A)+P(B)-P(AB)\qedhere
    \end{Equation}
\end{Proof}

\begin{BoxProperty}[逆事件的概率]
    对于任一事件$A$,总有
    \begin{Equation}
        P(\bar{A})=1-P(A)
    \end{Equation}
\end{BoxProperty}

\begin{Proof}
    根据\fancyref{def:对立事件},事件$A$和逆事件$\bar{A}$间满足
    \begin{Equation}
        A\cup\bar{A}=S\qquad
        A\bar{A}=\empty
    \end{Equation}
    即$A,\bar{A}$互不相容,故可运用\fancyref{def:可列可加性}
    \begin{Equation}
        P(S)=P(A\cup\bar{A})=P(A)+P(\bar{A})
    \end{Equation}
    而另外一方面,根据\fancyref{def:概率}中概率的规范性
    \begin{Equation}
        P(S)=1
    \end{Equation}
    因此
    \begin{Equation}*
        P(A)+P(\bar{A})=P(S)\qedhere
    \end{Equation}
\end{Proof}
\section{古典概型}
古典概型是概率论发展初期的主要研究对象,它的一些概念直观简单,具有广泛应用。

\subsection{古典概型的定义}
\begin{BoxDefinition}[古典概型]
    若随机试验$E$满足以下两个特征
    \begin{enumerate}
        \item 试验的样本空间仅包含有限个元素。
        \item 试验中的每个基本事件发生的可能性相同。
    \end{enumerate}
    则将随机试验$E$称为\uwave{古典概型}(Classical Probability)或\uwave{等可能概型}。
\end{BoxDefinition}

\subsection{古典概型的概率公式}
\begin{BoxFormula}[古典概型的概率公式]
    若事件$A$包含$k$个基本事件,而样本空间的大小为$n$,则
    \begin{Equation}
        P(A)=\frac{k}{n}
    \end{Equation}
\end{BoxFormula}
\begin{Proof}
    设试验的样本空间为$S=\qty{e_1,e_2,\cdots,e_n}$,根据\fancyref{def:古典概型}
    \begin{Equation}&[1]
        P(\qty{e_1})=
        P(\qty{e_2})=
        \cdots
        P(\qty{e_n})
    \end{Equation}
    由于基本事件是两两互不相容的,且$S=\qty{e_1}\cup\qty{e_2}\cup\cdots\cup\qty{e_n}$
    \begin{Equation}&[2]
        1=P(S)=P\qty(\BigCup[i=1][n]\qty{e_i})=\Sum[i=1][n]P(\qty{e_i})
    \end{Equation}
    将\xrefpeq{1}代入\xrefpeq{2},这里$i$可以是任意小于等于$n$的正整数
    \begin{Equation}&[3]
        1=nP(\qty{e_i})
    \end{Equation}
    故
    \begin{Equation}
        P(\qty{e_i})=\frac{1}{n}
    \end{Equation}
    而事件$A$中包含$k$个基本试验,即$A=\qty{e_{a1},e_{a2},\cdots,e_{ak}}$
    \begin{Equation}*
        P(A)=\Sum[i=1][k]P(\qty{e_{ai}})=\Sum[i=1][k]\frac{1}{n}=\frac{k}{n}\qedhere
    \end{Equation}
\end{Proof}