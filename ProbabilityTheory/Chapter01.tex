\chapter{概率论的基本概念}
自然界中发生的现象是多种多样的,有这样一类现象,在一定条件下必然会发生,例如,向上抛一物体必然下落,同性电荷必然相互排斥,异性电荷必然相互吸引,等等,这类现象称为\uwave{确定性现象}(Deterministic Phenomenon)。但是,还有这样一类现象,例如,在相同条件下抛同一枚硬币,其落下后,可能朝上,可能朝下。这里现象,在一定的条件下,可能出现这样的结果,可能出现那样的结果,但是经过长期实践并深入研究后,发现这类现象在大量重复试验或观察下,它的结果却呈现出某种规律性,抛硬币得到正面朝上和反面朝上的情况数大概都是一半。这种在大量重复试验或观察中所呈现出的固有规律性,就是我们之后所说的\uwave{统计规律性}(Statistical Regularity)。综上,\empx{在个别试验中结果呈现出不确定性,在大量重复试验中结果又具有统计规律性的现象},我们称之为\uwave{随机现象}(Random Phenomenon)。而所谓\uwave{概率论}(Probability Theory)与\uwave{数理统计}(Mathematical Statistics),即研究统计规律的数学。

\section{随机试验}
我们遇到过各种试验,在这里,我们把试验作为一个含义广泛的术语,例如
\begin{itemize}
    \item $E_1$:将一枚硬币抛掷一次,观察正面$H$和反面$T$出现的情况。
    \item $E_2$:将一枚硬币抛掷三次,观察正面$H$和反面$T$出现的情况。 
    \item $E_3$:抛一颗骰子,观察出现的点数。
    \item $E_4$:在一批灯泡中任意抽取一只,测试它的极限寿命(以小时为单位)。
\end{itemize}
我们这里举了四个试验的例子,它们有一些共同的特点,我们总结如下
\begin{BoxDefinition}[随机试验]
    \uwave{随机试验}(Experiment)是指具有下述三个特点的试验,常记为$E$
    \begin{enumerate}
        \item 可以在相同的条件下重复进行。
        \item 试验的可能结果不止一个,并且事先能明确试验的所有可能结果。
        \item 试验进行之前,我们无法确定哪一个结果会出现
    \end{enumerate}
\end{BoxDefinition}


\section{随机事件与样本空间}

\subsection{样本空间和样本点}
关于\fancyref{def:随机试验}中提到的一个随机试验,尽管在每次试验前并不能预知试验的结果,但试验的所有可能组成的集合是已知的,这种集合,可以是有限集,可以是无限集。
\begin{BoxDefinition}[样本空间]
    定义随机试验$E$每个可能的结果组成的集合$S$,为$E$的\uwave{样本空间}(Sample Space)
    \begin{Equation}
        S=\qty{e_1,e_2,\cdots,e_i,\cdots}
    \end{Equation}
\end{BoxDefinition}
\begin{BoxDefinition}[样本点]
    定义随机试验$E$的每个结果$e_i$,为$E$的\uwave{样本点}(Sample Point)。
\end{BoxDefinition}

例如\xref{sec:随机试验}中列出的试验$E_1,E_2,E_2,E_3$的样本空间就是
\begin{Gather}*[6pt]
    S_1:\qty{H,T}\\
    S_2:\qty{HHH,HHT,HTH,HTT,THH,THT,TTH,TTT}\\
    S_3:\qty{1,2,3,4,5,6}\\
    S_4:\qty{t\mid t\geq 0}
\end{Gather}
而上述样本空间的集合中的每个元素,就是样本点。

\subsection{随机事件}
在实际中,当进行随机试验时,我们常常关心某些样本点所组成的集合。
\begin{BoxDefinition}[随机事件]
    \uwave{随机事件}(Event)是随机试验$E$的样本空间$S$的一个子集,常记为$A$

    在每次试验$E$中,当且仅当该子集$A$中一个样本点发生,我们称这一事件$A$发生。
\end{BoxDefinition}

例如,关于\xref{sec:随机试验}中掷骰子的随机试验$E_3$
\begin{itemize}
    \item 事件$A_1$是掷出点数$2$,则$A_1=\qty{2}$
    \item 事件$A_2$是掷出偶数点数,则$A_2=\qty{2,4,6}$
\end{itemize}
特别的,我们赋予那些仅包含一个样本点的事件一个特别的名称,即基本事件。
\begin{BoxDefinition}[基本事件]
    \uwave{基本事件}(Elementary Event)是指仅包含一个样本点的事件。
\end{BoxDefinition}
\begin{itemize}
    \item 掷骰子的试验$E_3$中,基本事件有六个,有$\qty{1},\qty{2},\qty{3},\qty{4},\qty{5},\qty{6}$
    \item 掷硬币的试验$E_1$中,基本事件有两个,有$\qty{H},\qty{T}$
\end{itemize}

% 除此之外,还有两个特别的事件需要定义
\begin{BoxDefinition}[必然事件]
    定义\uwave{必然事件}(Certain Event)为样本空间$S$自身,它是$S$的子集。
\end{BoxDefinition}
\begin{BoxDefinition}[不可能事件]
    定义\uwave{不可能事件}(Impossible Event)为空集$\empty$,它是$S$的子集。
\end{BoxDefinition}

\subsection{随机事件的关系与运算}
我们说,\empx{事件是一个集合},因此,事件间的关系和运算可以按照集合论中集合之间的关系和运算来处理,下面,我们将根据“事件发生”的含义,给出这些运算在概率论中的具体含义。

\begin{BoxDefinition}[事件的包含]
    若$A\subset B$,则称事件$B$包含事件$A$,即事件$A$发生必然导致事件$B$发生。
\end{BoxDefinition}

\begin{BoxDefinition}[事件的相等]
    若$A\subset B$又有$B\subset A$,则称事件$B$等于事件$A$,记为$A=B$。
\end{BoxDefinition}

\begin{BoxDefinition}[和事件]
    定义$A,B$的\uwave{和事件}(Union of Events)
    \begin{Equation}
        A\cup B=\qty{x\mid x\in A~\text{or}~x\in B}
    \end{Equation}
    其意义是,当且仅当$A,B$至少有一个发生时,和事件$A\cap B$发生。
\end{BoxDefinition}

\begin{BoxDefinition}[积事件]*
    定义$A,B$的\uwave{积事件}(Intersection of Events)
    \begin{Equation}
        A\cap B=AB=\qty{x\mid x\in A~\text{and}~x\in B}
    \end{Equation}
    其意义是,当且仅当$A,B$同时发生时,积事件$A\cup B$发生。
\end{BoxDefinition}

\begin{BoxDefinition}[差事件]
    定义$A,B$的\uwave{差事件}(Set Difference)
    \begin{Equation}
        A-B=\qty{x\mid x\in A~\text{and}~x\notin B}
    \end{Equation}
    其意义是,当且仅当$A$发生且$B$不发生,差事件$A-B$发生。
\end{BoxDefinition}

\begin{BoxDefinition}[互斥事件]
    若事件$A.B$满足下式,则$A,B$是\uwave{互斥}的
    \begin{Equation}
        A\cap B=\empty
    \end{Equation}
\end{BoxDefinition}

\begin{BoxDefinition}[对立事件]
    若事件$A.B$满足下式,则$A,B$是\uwave{对立}的
    \begin{Equation}
        A\cap B=\empty\qquad A\cup B=S
    \end{Equation}
\end{BoxDefinition}
很明显,对立事件是互斥事件的一种特殊情况,对立必然互斥,但对立还要求“除了$A,B$没有更多情况”,因此,我们可以将$A$唯一的那个对立事件记为$\bar{A}$,显然其满足$\bar{A}=S-A$。
