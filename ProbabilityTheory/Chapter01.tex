\chapter{概率论的基本概念}
自然界中发生的现象是多种多样的,有这样一类现象,在一定条件下必然会发生,例如,向上抛一物体必然下落,同性电荷必然相互排斥,异性电荷必然相互吸引,等等,这类现象称为\uwave{确定性现象}(Deterministic Phenomenon)。但是,还有这样一类现象,例如,在相同条件下抛同一枚硬币,其落下后,可能朝上,可能朝下。这里现象,在一定的条件下,可能出现这样的结果,可能出现那样的结果,但是经过长期实践并深入研究后,发现这类现象在大量重复试验或观察下,它的结果却呈现出某种规律性,抛硬币得到正面朝上和反面朝上的情况数大概都是一半。这种在大量重复试验或观察中所呈现出的固有规律性,就是我们之后所说的\uwave{统计规律性}(Statistical Regularity)。综上,\empx{在个别试验中结果呈现出不确定性,在大量重复试验中结果又具有统计规律性的现象},我们称之为\uwave{随机现象}(Random Phenomenon)。而所谓\uwave{概率论}(Probability Theory)与\uwave{数理统计}(Mathematical Statistics),即研究统计规律的数学。

\section{随机试验}
我们遇到过各种试验,在这里,我们把试验作为一个含义广泛的术语,例如
\begin{itemize}
    \item $E_1$:将一枚硬币抛掷一次,观察正面$H$和反面$T$出现的情况。
    \item $E_2$:将一枚硬币抛掷三次,观察正面$H$和反面$T$出现的情况。 
    \item $E_3$:抛一颗骰子,观察出现的点数。
    \item $E_4$:在一批灯泡中任意抽取一只,测试它的极限寿命(以小时为单位)。
\end{itemize}
我们这里举了四个试验的例子,它们有一些共同的特点,我们总结如下
\begin{BoxDefinition}[随机试验]
    \uwave{随机试验}(Experiment)是指具有下述三个特点的试验,常记为$E$
    \begin{enumerate}
        \item 可以在相同的条件下重复进行。
        \item 试验的可能结果不止一个,并且事先能明确试验的所有可能结果。
        \item 试验进行之前,我们无法确定哪一个结果会出现
    \end{enumerate}
\end{BoxDefinition}

