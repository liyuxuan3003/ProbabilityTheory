\section{随机样本}
在数理统计中,我们往往研究有关对象的某一项数量指标。为此,我们考虑与这一数量指标相联系的随机试验,并对这一数量指标进行试验和观察。我们将试验全部可能的观察值称为\uwave{总体}(Statistical Population),其中每一个可能的观察值称为\uwave{个体}(Individual),并且
\begin{itemize}
    \item 总体中所包含的个体的个数称为总体的\uwave{容量}。
    \item 容量有限的总体,称为\uwave{有限总体}。
    \item 容量无限的总体,称为\uwave{无限总体}。
\end{itemize}
总体中的每一个个体是随机试验的一个观察值,因此它是某一随机变量$X$的值,这样,一个总体就对应于一个随机变量$X$。我们对总体的研究,其实就是对一个随机变量$X$的研究。

在今后,我们将不区分总体与总体相应的随机变量$X$,笼统的称为总体$X$。

总体的分布通常是未知的,在数理统计中,我们会从总体中抽取一部分个体,根据获得的数据来对总体分布作出推断。我们将从总体中抽出的部分个体称为总体的一个\uwave{样本}(Sample)。而所谓从总体中抽取一部分个体,就是对总体$X$依次进行观察并记录其结果。我们在相同条件下对总体$X$进行$n$次独立重复的观察,将$n$次试验的次序依次记为$X_1,X_2,\cdots,X_n$,它们都是与$X$同分布的随机变量,这样得到的$X_1,X_2,\cdots,X_n$称为来自总体$X$的一个简单随机样本。而当$n$次观察一经完成,我们就得到了一组实数的观察值$x_1,x_2,\cdots,x_n$,称为\uwave{样本值}。
\begin{BoxDefinition}[样本]
    设$X$是具有分布函数$F$的随机变量,若$X_1,X_2,\cdots,X_n$是具有同一分布函数$F$的相互独立的随机变量,则称$X_1,X_2,\cdots,X_n$为总体$X$的容量为$n$的简单随机样本,简称为样本。而样本的观察值$x_1,x_2,\cdots,x_n$则称为样本值,或称为$X$的$n$个独立的观察值。
\end{BoxDefinition}

