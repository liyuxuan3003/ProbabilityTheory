\section{两个随机变量的函数的分布}
在\xref{sec:随机变量的函数的分布}中,已经探讨了一个而随机变量的函数的分布,本节将讨论两个随机变量构成的函数的分布,然而,这一般而言是比较困难的,因此我们仅会就几个特殊情况进行分析。

\subsection{随机变量和的分布}
\begin{BoxFormula}[随机变量和的分布]
    设$(X,Y)$是二维连续型随机变量,具有概率密度$f(x,y)$,则
    \begin{Equation}
        Z=X+Y
    \end{Equation}
    仍然为连续型随机变量,其概率密度为
    \begin{Equation}
        f_{X+Y}(z)=\Int[-\infty][\infty]f(z-y,y)\dd{y}=\Int[-\infty][\infty]f(x,z-x)\dd{x}
    \end{Equation}
    特别的,若$X,Y$是独立的,则可以简化为
    \begin{Equation}
        \qquad\qquad
        f_{X+Y}(z)=f_{X}*f_{Y}=\Int[-\infty][\infty]f_X(z-y)f_Y(y)\dd{y}=\Int[-\infty][\infty]f_X(x)f_Y(z-x)\dd{x}
        \qquad\qquad
    \end{Equation}
    这里$f_X*f_Y$代表$f_X$和$f_Y$的卷积。
\end{BoxFormula}

\begin{Proof}
    首先,我们来求$Z=X+Y$的分布函数$F_Z(z)$,即有
    \begin{Equation}&[1]
        F_Z(z)=P\qty{Z\leq z}=P\qty{X+Y\leq z}=\Isnt[G]f(x,y)\dx\dy
    \end{Equation}
    这里的积分区域$G$是直线$x+y=z$及其左下半平面
    \begin{Equation}&[2]
        G: x+y\leq z
    \end{Equation}
    由此,\xrefpeq{1}中的二重积分可以转换为以下累次积分
    \begin{Equation}
        F_Z(z)=\Int[-\infty][\infty]\qty[\Int[-\infty][z-y]f(x,y)\dx]\dy
    \end{Equation}
    如果希望由$F_Z(z)$得到$f_Z(z)$,需要让$F_Z(z)$的表达式的最外层是$-\infty$至$z$的积分。

    而为了达成这个目的,令$x=u-y$,则$\dx=\dd{u}$,且当$x=z-y$时$u=z$,故
    \begin{Equation}
        F_Z(z)=\Int[-\infty][\infty]\qty[
        \Int[-\infty][z]f(u-y,y)\dd{u}]\dy
    \end{Equation}
    积分顺序是可以交换的
    \begin{Equation}
        F_Z(z)=\Int[-\infty][z]\qty[\Int[-\infty][\infty]f(u-y,y)\dy]\dd{u}
    \end{Equation}
    根据\fancyref{def:概率密度}
    \begin{Equation}
        F_Z(z)=\Int[-\infty][z]f_Z(u)\dd{u}
    \end{Equation}
    因此
    \begin{Equation}
        f_Z(u)=\Int[-\infty][\infty]f(u-y,y)\dy
    \end{Equation}
    这里将$u$改写为$z$不会有什么妨碍
    \begin{Equation}
        f_Z(z)=\Int[-\infty][\infty]f(z-y,y)\dy
    \end{Equation}
    类似的,如果前面换元由$x$改为$y$,则可以得到
    \begin{Equation}
        f_Z(z)=\Int[-\infty][\infty]f(x,z-x)\dx
    \end{Equation}
    而特别的,如果$X,Y$独立的,则
    \begin{Equation}
        f(z-y,y)=f_X(z-y)f_Y(y)\qquad
        f(x,z-x)=f_X(x)f_Y(z-x)
    \end{Equation}
    此时积分式恰好符合$f_X$和$f_Y$的卷积,故亦可以记作$f_X*f_Y$。
\end{Proof}

\subsection{随机变量最大值和最小值的分布}
除了$Z=X+Y$,我们还比较关心$Z=\max\qty{X,Y}$和$Z=\min\qty{X,Y}$,即两个随机变量间的最大值和最小值的分布,但这里,我们的结论更倾向于用分布函数而非概率密度函数表示。
\begin{BoxFormula}[随机变量的最大值的分布]
    设$(X,Y)$是二维连续型随机变量,\textbf{且相互独立},具有分布函数$F(x,y)$,则
    \begin{Equation}
        Z=\max\qty{X,Y}
    \end{Equation}
    仍然为连续型随机变量,其分布函数为
    \begin{Equation}
        F_{\max}(z)=F_X(z)F_Y(z)
    \end{Equation}
\end{BoxFormula}
\begin{Proof}
    证明是容易的,由于$Z=\max\qty{X,Y}$
    \begin{Equation}
        F_{\max}(z)=P\qty{Z\leq z}=P\qty{X\leq z, Y\leq z}
    \end{Equation}
    根据\fancyref{def:独立性}
    \begin{Equation}*
        F_{\max}(z)=P\qty{X\leq z}P\qty{Y\leq z}=F_X(z)F_Y(z)\qedhere
    \end{Equation}
\end{Proof}

\begin{BoxFormula}[随机变量的最小值的分布]
    设$(X,Y)$是二维连续型随机变量,\textbf{且相互独立},具有分布函数$F(x,y)$,则
    \begin{Equation}
        Z=\min\qty{X,Y}
    \end{Equation}
    仍然为连续型随机变量,其分布函数为
    \begin{Equation}
        F_{\min}(z)=1-[1-F_X(z)][1-F_Y(z)]
    \end{Equation}
\end{BoxFormula}

\begin{Proof}
    证明也很容易,但最小值相较前面最大值有一点不同,由于$Z=\min\qty{X,Y}$,当$Z\leq z$时相当于是要求$X\leq z$或$Y\leq z$成立,而“或”在概率中不太好表达,故求其对立事件
    \begin{Equation}
        F_{\min}(z)=P\qty{Z\leq z}=1-P\qty{Z>z}=1-P\qty{X>z, Y>z}
    \end{Equation}
    根据\fancyref{def:独立性}
    \begin{Equation}*
        F_{\min}(z)=1-P\qty{X>z}P\qty{Y>z}=1-[1-F_X(z)][1-F_Y(z)]\qedhere
    \end{Equation}
\end{Proof}