\section{随机变量的函数的分布}
在实际中,我们往往会对某些随机变量的函数更感兴趣,例如,在一些试验中,我们所关心的随机变量往往不能直接测量得到,它是另外一个能测量的随机变量的函数。例如,测量的是圆柱的直径$d$,关心的是却是圆柱的底面积$\pi d^2/4$。在这一节中,我们将会讨论如何由已知的随机变量$X$的概率分布去求它的函数$Y=g(X)$的概率分布(这里$g(x)$是已知的连续函数)。

\begin{BoxTheorem}[随机变量的函数分布]
    设随机变量$X$具有概率密度$f_X(x)$,又设函数$g(x)$处处可导,且其导数$g'(x)$恒正或恒负,那么$Y=g(X)$仍是连续型随机变量,且其概率密度$f_Y(y)$为
    \begin{Equation}
        f_Y(y)=
        \begin{cases}
            f_X[g^{-1}(y)]\cdot\abs{g^{-1}(y)'},&\alpha<y<\beta\\
            0,&\text{otherwise}
        \end{cases}
    \end{Equation}
    其中,$\alpha=\min\qty{g(-\infty),g(\infty)}$,$\beta=\max\qty{g(-\infty),g(\infty)}$,$g^{-1}$是$g$的反函数。
\end{BoxTheorem}\goodbreak

\begin{Proof}\nopagebreak
    我们以$g'(x)>0$的情况为例证明,而$g'(x)<0$的情况是可以完全类似证明的。

    由于$g(x)$在$(-\infty,\infty)$内严格单调增加,故存在反函数$g^{-1}(y)$,且反函数在$(\alpha,\beta)$内严格单调增加。设$X,Y$的分布函数为$F_X(x), F_Y(y)$,现在,我们先来求$Y$的分布函数$F_Y(y)$。

    由于随机变量$Y=g(x)$仅在$\alpha,\beta$内取值
    \begin{itemize}
        \item 当$y\leq\alpha$时
        \begin{Equation}
            F_Y(y)=P\qty{Y\leq y}=0\qquad y\leq\alpha
        \end{Equation}
        \item 当$y\geq\beta$时
        \begin{Equation}
            F_Y(y)=P\qty{Y\leq y}=1\qquad y\geq\beta
        \end{Equation}
        \item 当$\alpha<y<\beta$时
        \begin{Equation}
            F_Y(y)=P\qty{Y\leq y}=P\qty{g(X)\leq y}=P\qty{X\leq g^{-1}(y)}=F_X[g^{-1}(y)]
        \end{Equation}
    \end{itemize}
    因此
    \begin{Equation}
        F_Y(y)=\begin{cases}
            0,&y\leq\alpha\\
            F_X[g^{-1}(y)],&\alpha<y<\beta\\
            1,&y\geq\beta
        \end{cases}
    \end{Equation}
    根据\fancyref{def:概率密度},运用复合函数的求导法则
    \begin{Equation}
        f_Y(y)=F_Y'(y)=
        \begin{cases}
            f_X[g^{-1}(y)]\cdot g^{-1}(y)',&\alpha<y<\beta\\
            0,&\text{otherwise}
        \end{cases}
    \end{Equation}
    而对于$g'(x)<0$的情况,上式中$g^{-1}(y)'$会变为$-g^{-1}(y)'$,故可统一记为$\abs{g^{-1}(y)'}$。
\end{Proof}

\begin{BoxProperty}[正态分布的线性变换]
    设随机变量$X\sim N(\mu,\sigma^2)$,则$X$的线性函数
    \begin{Equation}
        Y=aX+b
    \end{Equation}
    亦服从正态分布,且$Y\sim N(a\mu+b,(a\sigma)^2)$。
\end{BoxProperty}

\begin{Proof}
    根据\fancyref{def:正态分布},随机变量$X$的概率密度为
    \begin{Equation}&[1]
        f_X(x)=\frac{1}{\sqrt{2\pi}\sigma}\exp[-\frac{(x-\mu)^2}{2\sigma^2}]
    \end{Equation}
    现在已知$g(x)=y=ax+b$,很容易求出反函数
    \begin{Equation}&[2]
        g^{-1}(y)=x=\frac{y-b}{a}
    \end{Equation}
    其导数为
    \begin{Equation}&[3]
        g^{-1}(y)=-\frac{1}{a}
    \end{Equation}
    因此,根据\fancyref{thm:随机变量的函数分布}
    \begin{Equation}&[4]
        f_Y(y)=\frac{1}{\abs{a}}f_X\qty(\frac{y-b}{a})
    \end{Equation}
    在\xrefpeq{4}中代入\xrefpeq{1}
    \begin{Equation}
        f_Y(y)=\frac{1}{\abs{a}}\frac{1}{\sqrt{2\pi}\sigma}\exp[-\frac{[(y-b)/a-\mu]^2}{2\sigma^2}]
    \end{Equation}
    在分式上下同时乘$a^2$
    \begin{Equation}
        f_Y(y)=\frac{1}{\sqrt{2\pi}\abs{a}\sigma}\exp[-\frac{[y-(a\mu+b)]^2}{2(a\sigma)^2}]
    \end{Equation}
    即有$Y\sim N(a\mu+b,(a\sigma)^2)$。
\end{Proof}

特别的,如果取$a=1/\sigma$,取$b=-\mu/\sigma$,则有
\begin{Equation}
    Y=\frac{1}{\sigma}X-\frac{\mu}{\sigma}=\frac{X-\mu}{\sigma}\qquad Y\sim N(0,1)
\end{Equation}

这就是\xref{subsec:正态分布}中\fancyref{lem:标准正态分布与正态分布}的结果。